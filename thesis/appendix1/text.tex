% Text for appendix 1
\chapter{fiasco: A Python Interface to the CHIANTI Atomic Database}\label{ap:fiasco}

% spell-checker: disable %
\begin{pycode}[appendix1]
name = 'appendix1'
ap1 = texfigure.Manager(
    pytex,
    os.path.join('.', name),
    number=9,
    **{k: os.path.join('.', name, v) for k,v in manager_opts.items()}
)
\end{pycode}
% spell-checker: enable %

As noted in \autoref{sec:chianti}, the CHIANTI atomic database is an invaluable resource in the field of solar physics, both for modeling and interpreting observations. During the course of my graduate work, I developed the fiasco\footnote{The name of the package derives from the Italian word \textit{fiasco}, the style of bottle typically used to serve a wine from the Chianti region of Italy.} package for interacting with CHIANTI in the Python programming language. fiasco provides an intuitive Python interface to every part of the CHIANTI database as well as many calculations for many common derived quantities. The goal of the fiasco package is to provide largely the same functionality as the CHIANTI IDL routines via an object-oriented interface \citep[similar to ChiantiPy][]{landi_chiantiatomic_2012,barnes_chiantipy_2017} as well as improved interoperability with the greater Python ecosystems in astronomy and heliophysics. The following sections will briefly describe the fiasco software for parsing the raw atomic data and interacting with higher-level objects.

fiasco is developed openly on GitHub and fully documented\footnote{The full source is available at \href{https://github.com/wtbarnes/fiasco}{github.com/wtbarnes/fiasco} and the documentation, which is rebuilt at each code check-in, is available at \href{https://fiasco.readthedocs.io/en/latest/}{fiasco.readthedocs.io}.}, including several examples. fiasco also includes a test suite that is run at each code check-in to prevent unexpected changes. In the following sections, I provide a brief overview of the package. Additional details can be found in the documentation.  

\section{Parsing Data}\label{sec:parsing-chianti}

For each ion in the CHIANTI database, there are a number of different filetypes representing the different pieces of information attached to each ion (e.g. energy levels, transition wavelengths, thermally averaged collision strengths). While all of these files are in readable plaintext format, the layout of the data in each file is slightly different. Additionally, no units or descriptions of the data are provided in the file.

fiasco provides an easy to use file parser that takes the name of any CHIANTI filename in the database and parses it into the commonly-used Astropy \pyv{Table} format. As an example, to read the data from the energy level file for Fe XVI,
% spell-checker: disable %
\begin{pyverbatim}[appendix1][baselinestretch=1,xleftmargin=3em]
from fiasco.io import Parser
p = Parser('fe_16.elvlc')
table = p.parse()
\end{pyverbatim}
% spell-checker: enable %
\autoref{tab:appendix1:fe16elvlc} shows the \LaTeX{} representation of the table directly parsed from the file by fiasco, including units. To access the observed energies of each level of Fe XVI and convert them to \si{\erg},
% spell-checker: disable %
\begin{pyverbatim}[appendix1][baselinestretch=1,xleftmargin=3em]
e_obs = table['E_obs']
e_obs_erg = (e_obs * const.h * const.c).to(u.erg)
\end{pyverbatim}
% spell-checker: enable %
The data in each column is an Astropy \pyv{u.Quantity} and has units attached to it as appropriate. Additional metadata, including descriptions of each of the columns and the information in the footer of the raw data file can be accessed via the \pyv{table.meta} attribute.

% spell-checker: disable %
\begin{pycode}[appendix1]
from fiasco.io import Parser
p = Parser('fe_16.elvlc')
table = p.parse()
with io.StringIO() as f:
    table2 = copy.deepcopy(table)
    table2.rename_column('E_obs', r'$E_\textup{obs}$')
    table2.rename_column('E_th', r'$E_\textup{th}$')
    table2.rename_column('L_label', r'$L$')
    table2.rename_column('J', r'$J$')
    astropy.io.ascii.write(
        table2[:5],
        format='latex',
        caption=r'The first 5 rows of the energy level file for Fe XVI, \texttt{fe\_16.elvlc}. Additional metadata and units are available in the metadata of the \pyv|Table| object.\label{tab:appendix1:fe16elvlc}',
        output=f,
        latexdict = { 'data_start':r'\midrule', 'data_end': r'\bottomrule',
                      'header_start': r'\toprule', 'col_align': 'cccccccc',
                      'preamble':r'\begin{center}', 'tablefoot':r'\end{center}'
                    }
    )
    tab = f.getvalue()
\end{pycode}
\py[appendix1]|tab|
% spell-checker: enable %

Other filenames can be parsed in the exact same manner. Because each file format (e.g. \texttt{wgfa}, \texttt{elvlc}) is slightly different, a unique parser subclass is implemented for each filetype in the CHIANTI database. There are 23 unique filetypes. Rather than forcing the user to call a different function or instantiate a separate class for each unique filetype, fiasco uses a \textit{factory pattern}, a common design pattern in software engineering, to create an instance of a class ``on the fly'' based on the filename passed to the \pyv{Parser} object. This greatly simplifies the user interface by providing a single entrypoint for parsing data files.

A key advantage of fiasco is that it rebuilds the entire CHIANTI database as a hierarchical data format (HDF5) file. While ChiantiPy and the CHIANTI IDL routines read the raw data files directly, fiasco parses the raw data files once and then writes the entire database to a single HDF5 file. This is more efficient, particular in the case of large files (e.g. Fe IX and XI with many transitions), as HDF5 allows efficient read access to parts of the database without reading unneeded data into memory. 

\section{The \texttt{Ion} Class}\label{sec:fiasco-ion}

Similar to the ChiantiPy package, fiasco provides several useful abstractions of the database. The first is the \pyv{Ion} object. To instantiate an ion of Fe XVI,
% spell-checker: disable %
\begin{pyverbatim}[appendix1][baselinestretch=1,xleftmargin=3em]
from fiasco import Ion
T = np.logspace(4, 9, 100) * u.K
fe16 = Ion('Fe 16', temperature=T)
# alternatively
fe16 = Ion('iron 16', temperature=T)
\end{pyverbatim}
% spell-checker: enable %
Note that, just as in the synthesizAR package (see \autoref{ch:synthesizar}), all inputs corresponding to physical quantities require units.

Basic metadata can be accessed as attributes on the ion,
% spell-checker: disable %
\begin{pyverbatim}[appendix1][baselinestretch=1,xleftmargin=3em]
fe16.ion_name
fe16.element_name
fe16.atomic_symbol
fe16.abundance
\end{pyverbatim}
% spell-checker: enable %
Additionally, information about the energy levels can be accessed by indexing the object directly as a list
% spell-checker: disable %
\begin{pyverbatim}[appendix1][baselinestretch=1,xleftmargin=3em]
fe16[0] # the first energy level, a Level object
fe16[1].level
fe16[1].energy.to(u.eV)
\end{pyverbatim}
% spell-checker: enable %
Note that the result of indexing the \pyv{Ion} object is another object corresponding to the energy level within an ion which has several attributes attached to it. Additionally, the raw data files for each ion can also be accessed via attributes,
% spell-checker: disable %
\begin{pyverbatim}[appendix1][baselinestretch=1,xleftmargin=3em]
fe16._elvlc # energy levels
fe16._wgfa # transitions 
\end{pyverbatim}
% spell-checker: enable %
In general, these are reserved for internal use by fiasco such that the user need not worry about naming conventions of CHIANTI file formats.

Perhaps most importantly, fiasco provides simple interface for computing derived quantities from the atomic data attached to each ion. For example, to calculate the total ionization and recombination rates as a function of temperature for Fe XVI,
% spell-checker: disable %
\begin{pyverbatim}[appendix1][baselinestretch=1,xleftmargin=3em]
i_rate = fe16.ionization_rate()
r_rate = fe16.recombination_rate()
\end{pyverbatim}
% spell-checker: enable %
Many other derived quantities are available. A complete list is available in the documentation.

\section{The \texttt{Element} Class}\label{sec:fiasco-element}

In addition to an individual ion, fiasco also provides an abstraction for many ions of the same element, the \pyv{Element} class. To implement an \pyv{Element} object for iron,
% spell-checker: disable %
\begin{pyverbatim}[appendix1][baselinestretch=1,xleftmargin=3em]
from fiasco import Element
fe = Element('iron', temperature=T)
\end{pyverbatim}
% spell-checker: enable %
Basic metadata about the element is available via attributes of the class instance,
% spell-checker: disable %
\begin{pyverbatim}[appendix1][baselinestretch=1,xleftmargin=3em]
fe.element_name
fe.atomic_number
fe.atomic_symbol
fe.abundance
\end{pyverbatim}
% spell-checker: enable %

Additionally, just as the \pyv{Ion} object could be indexed to retrieve the energy levels, the \pyv{Element} object can be indexed to retrieve the constituent ions,
% spell-checker: disable %
\begin{pyverbatim}[appendix1][baselinestretch=1,xleftmargin=3em]
fe[0] # returns Ion object for Fe 1
fe[15] # returns Ion object for Fe 16
\end{pyverbatim}
% spell-checker: enable %

The \pyv{Element} object can also be used to compute quantities not defined for an individual ion such as the equilibrium population fractions (see \autoref{sec:ioneq}),
% spell-checker: disable %
\begin{pyverbatim}[appendix1][baselinestretch=1,xleftmargin=3em]
ioneq = fe.equilibrium_ionization()
\end{pyverbatim}
% spell-checker: enable %

\section{Working with Multiple Ions}\label{sec:ion-collection}

Some quantities such as spectra or radiative loss curves require data from more than one ion and not necessarily of the same element. fiasco provides a general \pyv{IonCollection} object.An ion collection can be created directly,
% spell-checker: disable %
\begin{pyverbatim}[appendix1][baselinestretch=1,xleftmargin=3em]
from fiasco import IonCollection
col = IonCollection(Ion('Fe 5', T), Ion('Fe 13', T))
\end{pyverbatim}
% spell-checker: enable %
or by literally adding the objects together
% spell-checker: disable %
\begin{pyverbatim}[appendix1][baselinestretch=1,xleftmargin=3em]
col = Ion('Fe 5', T) + Ion('Fe 13', T)
\end{pyverbatim}
% spell-checker: enable %
The latter is implemented by overriding the addition operator on the \pyv{Ion} and \pyv{IonCollection} classes. This type of notation allows for an intuitive exploration of how adding or removing ions impacts spectra or other quantities like the radiative loss curve (\autoref{sec:rad-loss}).

Note that the \pyv{Element} class is just a subclass of \pyv{IonCollection} where every ion is from the same element. The \pyv{EmissionModel} class (\autoref{sec:atomic-physics}) in the synthesizAR forward modeling code is also a subclass of \pyv{IonCollection}.