% Text for chapter 1
\chapter{Introduction}\label{ch:introduction}

% Figure manager for Chapter 1
% spell-checker: disable
%\begin{pycode}[chapter1]
%name = 'chapter1'
%ch1 = texfigure.Manager(
%    pytex,
%    os.path.join('.', name),
%    number=1,
%    **{k: os.path.join('.', name, v) for k,v in manager_opts.items()}
%)
%\end{pycode}
% spell-checker: enable

\section{The Structure of the Solar Atmosphere}

\section{A History of Solar Observations}

\section{The Solar Magnetic Field}

\subsection{Flux Emergence}

% dynamo, flux emergence, formation of ARs (define AR)

\subsection{Reconnection}

\subsection{Observations}

% comments on how we observe magnetic fields in the corona (Zeeeman, Hanle)
% LOS vs vector magnetograms
% show a synoptic map (GONG) or full-disk (HMI)

\subsection{Field Extrapolation}

% should we introduce MHD equations here?

\textit{Magnetic field extrapolation} is a commonly-used technique for approximating the three-dimensional vector magnetic field in the corona given an observed magnetogram. Following treatment in \citet[Chapter 3]{priest_magnetohydrodynamics_2014}, the ideal magnetohydrodynamic (MHD) momentum equation in magnetohydrostatic balance can be written as,
\begin{equation}\label{eq:magnetohydrostatic_balance}
    0 = -\nabla p + \frac{1}{c}\mathbf{j}\times\mathbf{B}
\end{equation}
where $p$ is the pressure, $\mathbf{j}$ is the current density, and $\mathbf{B}$ is the magnetic field. In a low-$\beta$ plasma where the magnetic pressure dominates over the thermal pressure (see \autoref{ch:loops}), the first term on the right-hand side can often be neglected such that \autoref{eq:magnetohydrostatic_balance} becomes,
\begin{equation}\label{eq:force_free}
    \mathbf{j}\times\mathbf{B} = 0.
\end{equation}
\autoref{eq:force_free} is the so-called \textit{force-free} condition. Given Amp\'{e}re's law from Maxwell's equations for electrodynamics,
\begin{equation}\label{eq:ampere}
    \nabla\times\mathbf{B} = \frac{4\pi}{c}\mathbf{j},
\end{equation}
this implies that,
\begin{equation}\label{eq:ampere_force_free}
    \nabla\times\mathbf{B} = \alpha\mathbf{B},
\end{equation}
where, in general, the scalar $\alpha$ may be some function of position $\mathbf{r}$.

In the case of $\alpha=0$, $\mathbf{j}=0$ (from \autoref{eq:ampere}) and the magnetic field is said to be current-free or \textit{potential}. \autoref{eq:ampere_force_free} implies that $\mathbf{B}$ is also curl-free such that it can be expressed as $\mathbf{B}=\nabla\phi$, where $\phi$ is some scalar potential. Combining this expression with the requirement from Maxwell's equations that the magnetic field must always be divergence-free ($\nabla\cdot\mathbf{B}=0$) gives Laplace's equation,
\begin{equation}\label{eq:laplace}
    \nabla^2\phi = 0,
\end{equation}
where $\nabla^2$ is the Laplacian operator. If the normal component of the magnetic field is specified on the lower boundary (e.g. from a photospheric LOS magnetogram), the solution within a closed volume is unique \citep{priest_magnetohydrodynamics_2014}.

Several methods have been developed to solve \autoref{eq:laplace} for the coronal magnetic field. The potential field source surface (PFSS) model of \citet{schatten_model_1969} solves \autoref{eq:laplace} for the global corona given a synoptic photospheric magnetogram as the lower boundary input and under the assumption that the field is purely radial at some ``source surface'', typically $2.5R_{\solar}$. Additionally, the Green's function method of \citet{schmidt_observable_1964} can be used to efficiently determine the potential magnetic field on the scale of a single \AR{} on a Cartesian grid given a LOS magnetogram. \autoref{sec:potential_field} will describe the latter method in detail. While the work presented in this thesis will only make use of photospheric LOS magnetogram data, there exist many techniques for computing field extrapolations from vector magnetograms as well \citep[see review by][]{welsch_deriving_2016}.

The potential field represents the lowest possible energy state of the magnetic field and is likely to be an appropriate approximation provided the magnetic energy dominates over the thermal energy ($\beta<1$) and the field has had sufficient time to relax to the lowest energetic state \citep{priest_magnetohydrodynamics_2014}. Thus, a field with a non-zero current is in a higher energy state than a potential field. From \autoref{eq:ampere}, if $\mathbf{j}\neq0$ then $\alpha\neq0$. Provided \autoref{eq:force_free} holds, solutions to \autoref{eq:ampere_force_free} represent non-potential force-free fields and in general are much more difficult to compute than potential field solutions. If $\alpha$ is constant, the solution is a linear force-free field, but if $\alpha$ is a function of a position $\mathbf{r}$, the magnetic field is said to be non-linear force-free. See \citet{wiegelmann_solar_2012} for a comprehensive review of force-free magnetic fields in solar physics. 

\section{Heating Processes in the Corona}

\subsection{Waves versus Reconnection}

\subsection{Nanoflare Heating}

\section{Aims of this Work}

% Nomenclature
\nomenclature[a-rsun]{$R_{\solar}$}{radius of the Sun, $\approx\SI{6.957e10}{\cm}$}
\nomenclature[z-ar]{AR}{active region}
\nomenclature[z-tr]{TR}{transition region}
\nomenclature[z-mhd]{MHD}{magnetohydrodynamics}
\nomenclature[z-noaa]{NOAA}{National Oceanic and Atmospheric Administration}
