% Text for chapter 1
\chapter{The Physics of Coronal Loops}\label{ch:loops}

% Figure manager for Chapter 2
% spell-checker: disable
%\begin{pycode}[chapter2]
%name = 'chapter2'
%ch1 = texfigure.Manager(
%    pytex,
%    os.path.join('.', name),
%    number=2,
%    **{k: os.path.join('.', name, v) for k,v in manager_opts.items()}
%)
%\end{pycode}
% spell-checker: enable

% conceptual comments on loops
% why is this treatment valid?
% maybe give MHD equations then say we dont need them

This is the chapter on coronal loops

\section{Hydrostatic Equilibrium}

%Introduce hydrostatic equations

%Talk about terms, what they mean

\subsection{The Isothermal Limit}

\subsection{Scaling Laws}\label{sec:scaling_laws}

%RTV, Serio, Martens

\subsection{Numerical Solutions}

%Show basic hydrostatic solutions via shooting method

\section{Hydrodynamics}

%show two-fluid hydro equations and how they relate to previously stated MHD equations
% discuss physics of each term

\subsection{The HYDRAD Model}

%Discuss HYDRAD in here, advantages, maybe a sample solution
% briefly discuss capabilities, numerical treatment

\subsection{The EBTEL Model}\label{sec:ebtel}

%Derive normal and two-fluid EBTEL equations, show examples

\subsubsection{Two-fluid Model}\label{sec:ebtel-two-fluid}

% Derivation of two-fluid model as given in second appendix of Barnes et al (2016a)
% details of numerical approach in ebtel++

\subsubsection{Single-fluid Model}\label{sec:ebtel-single-fluid}

% Show how two-fluid equations simplify to single-fluid model
% point out that this was original model, developed by Klimchuk, Cargill, et al.

\subsubsection{Modifications to $c_1$ During the Conductive Cooling Phase}\label{sec:c1-correction}

% First appendix in Barnes et al (2016a)
