% Text for chapter 1
\chapter{Emission Diagnostics of Coronal Heating}\label{ch:diagnostics}

% Figure manager for Chapter 3
\begin{pycode}[chapter_3]
name = 'chapter3'
ch1 = texfigure.Manager(
    pytex,
    os.path.join('.', name),
    number=3,
    **{k: os.path.join('.', name, v) for k,v in manager_opts.items()}
)
\end{pycode}

Diagnosing the properties of the underlying energy deposition in the corona is nontrivial as measurements are limited to remote sensing data from ground- and space-based instruments. Additionally, until the recently launched Parker Solar Probe mission \citep{fox_solar_2016}, \textit{in situ} measurements were limited to the solar wind at 1 AU. Thus, inferring the dynamics and energy budget of the coronal plasma necessitates the use of multiple diagnostics computed from the observed coronal emission at multiple wavelengths. In \autoref{sec:line_formation}, I discuss the physics of the formation of the optically thin solar spectrum. In \autoref{sec:dem}, I give an overview of the differential emission measure distribution (DEM) and discuss how the DEM can be used to understand the frequency of energy deposition in the solar atmosphere. Lastly, in \autoref{sec:timelag}, I discuss the timelag analysis technique and show how it can be used to visualize large-scale cooling patterns in active regions.

\nomenclature[z-au]{AU}{astronomical unit}
\nomenclature[z-dem]{DEM}{differential emission measure}

\section{Spectral Line Formation}\label{sec:line_formation}

The solar corona is \textit{optically thin}, meaning that all emitted photons are observed and that these photons are not absorbed or scattered between the emission site and detector. Because these photons travel uninterrupted, they provide a direct signature of the properties of the coronal plasma. One of the primary mechanisms for the formation of spectral emission lines in the solar corona is the spontaneous radiative decay of an electron in an excited state $j$ to a lower energy state $i$,
\begin{equation}\label{eq:radiative_decay}
    X_{k,j} \to X_{k,i} + h\nu_{ji},
\end{equation}
where $X_k$ is an ion of element $X$ in ionization stage $k$, $\nu_{ji}$ is the frequency of the atomic transition, and $h\nu_{ji}$ is the energy of the emitted photon.

The intensity of a spectral line for an atomic transition of wavelength $\lambda_{ji}=c/\nu_{ji}$, where $c$ is the speed of light in a vacuum, is given by,
\begin{equation}\label{eq:intensity}
    I(\lambda_{ji}) = \frac{1}{4\pi}\int\textup{d}h\,P(\lambda_{ji}),
\end{equation}
where the integration is taken along the line-of-sight (LOS) between the observer the emission site and $P(\lambda_{ji})$ is the \textit{emissivity}, or the radiative power per unit volume. The emissivity is given by,
\begin{equation}\label{eq:emissivity}
    P(\lambda_{ji}) = \frac{hc}{\lambda_{ji}}n_{X,k,j}A_{ji},
\end{equation}
where $n_{X,k,j}$ is the number density of $X_k$ ions in excited state $j$ and $A_{ji}$ is the probability of spontaneous emission, often referred to as the Einstein coefficient \citep{bradshaw_collisional_2013,del_zanna_solar_2018}.

In general, it is quite difficult to determine $n_{X,k,j}$, the density of ions in excited state $j$. As such, we can rewrite $n_{X,k,j}$ as,
\begin{align}\label{eq:nj}
    n_{X,k,j} &= \frac{n_{X,k,j}}{n_{X,k}}\frac{n_{X,k}}{n_X}\frac{n_X}{n_H}\frac{n_H}{n_e}n_e, \nonumber\\
              &= N_{X,k,j}f_{X,k}\textup{Ab}(X)\frac{n_H}{n_e}n_e,
\end{align}
where $n_e$ is the electron density, $\textup{Ab}(X)=n_X/n_H$ is the abundance of element $X$ relative to hydrogen, $n_H/n_e$ is the ratio of hydrogen ions to electrons, often approximated as $n_H/n_e\approx0.83$, $N_{X,k,j}=n_{X,k,j}/n_{X,k}$ is the level population of level $j$ or the fraction of $X_k$ ions in excited state $j$, and $f_{X,k}=n_{X,k}/n_X$ is the population fraction of ion $X_k$ \citep{del_zanna_solar_2018}. Plugging \autoref{eq:nj} into \autoref{eq:emissivity} yields a more convenient expression for the emissivity,
\begin{equation}\label{eq:emissivity_simplified}
    P(\lambda_{ji}) = 0.83\frac{hc}{\lambda_{ji}}\textup{Ab}(X)f_{X,k}N_{X,k,j}A_{ji}n_e.
\end{equation}
Both $\textup{Ab}(X)$ and $A_{ji}$, which is a function of electron temperature, $T_e$, can be looked up in the CHIANTI atomic database \citep{dere_chianti_1997,young_chianti_2016}. $N_{X,k,j}$ is a function of both $T_e$ and $n_e$ and can be computed by assuming the the excitation and deexcitation processes are in equilibrium. This is discussed in \autoref{subsec:collisional_excitation} and \autoref{subsec:level_pops}. $f_{X,k}$ is primarily a function of $T_e$ and is discussed in \autoref{subsec:ionization_recombination} and \autoref{subsec:ioneq_versus_nei}. Thus, for a given distribution of $T_e$ and $n_e$ along the LOS, we can compute the intensity of a transition $\lambda_{ji}$ using \autoref{eq:emissivity_simplified} and \autoref{eq:intensity}.

\nomenclature[a-h]{$h$}{Planck constant}
\nomenclature[a-c]{$c$}{speed of light in a vacuum}
\nomenclature[g-nu]{$\nu$}{photon frequency}
\nomenclature[s-k]{$k$}{ionization stage}
\nomenclature[z-los]{LOS}{line-of-sight}

\subsection{Collision Excitation of Atomic Levels}\label{subsec:collisional_excitation}

For a photon to be produced by the spontaneous radiative decay from excited state $j$ to lower energy state $i$ (see \autoref{eq:radiative_decay}), the ion must first be excited into state $j$. In the solar atmosphere, the most important excitation process is the inelastic collisions between ions and free electrons,
\begin{equation}\label{eq:inelastic_collisions}
    X_{k,i} + e(E_{\textup{initial}}) \to X_{k,j} + e(E_{\textup{final}})
\end{equation}

\hilite{Production of satellite lines via dielectronic recombination important for levels above the ionization potential}

\subsection{Level Populations}\label{subsec:level_pops}

\hilite{Write down level population calculation, discuss coronal model approximation}

\subsection{Ionization and Recombination}\label{subsec:ionization_recombination}

\subsection{Equilibrium versus Nonequilibrium Ionization}\label{subsec:ioneq_versus_nei}

\nomenclature[z-nei]{NEI}{nonequilibrium ionization}

\section{Continuum Emission}\label{sec:continuum}

\hilite{Not sure if this section is needed since we don't deal with continuum much though it is still an important mechanism}

\subsection{Free-free Emission}

\subsection{Free-bound Emission}

\section{The Differential Emission Measure Distribution}\label{sec:dem}

\section{Timelag Analysis}\label{sec:timelag}

Theory, show example with Gaussian pulses, discuss AIA channel sensitivity, relate timelag to cooling time

\section{Summary}
