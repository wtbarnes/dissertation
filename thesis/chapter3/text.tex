% Text for chapter 3
\chapter{Emission Mechanisms and Diagnostics of Coronal Heating}\label{ch:diagnostics}

% Figure manager for Chapter 3
% spell-checker: disable %
\begin{pycode}[chapter3]
name = 'chapter3'
ch3 = texfigure.Manager(
    pytex,
    os.path.join('.', name),
    number=3,
    **{k: os.path.join('.', name, v) for k,v in manager_opts.items()}
)
\end{pycode}
% spell-checker: enable %

Diagnosing the properties of the underlying energy deposition in the corona is nontrivial as measurements are limited to remote sensing data from ground- and space-based instruments. Until the recently launched Parker Solar Probe mission \citep{fox_solar_2016}, \textit{in situ} measurements were limited to the solar wind at 1 AU. Thus, inferring the dynamics and energy budget of the coronal plasma necessitates the use of multiple diagnostics computed from the observed coronal emission at multiple wavelengths. In this chapter, I will give an overview of how emission is produced in the corona and discuss observational diagnostics that can provide meaningful insight into how the plasma is heated. \autoref{sec:chianti} notes the practical importance of the CHIANTI database in modeling and interpreting solar observations. In \autoref{sec:line_formation} and \autoref{sec:continuum}, I review the physics of the formation of spectral lines and continuum radiation in the corona and \autoref{sec:aia_response} provides a detailed explanation of the temperature sensitivity of the AIA passbands. In the last two sections, I discuss two primary diagnostics for inferring properties of the energy deposition from observations: the differential emission measure distribution (\autoref{sec:dem}) and the time lag (\autoref{sec:timelag}).

\section{The CHIANTI Atomic Database}\label{sec:chianti}

The CHIANTI atomic database \citep{dere_chianti_1997,young_chianti_1998,landi_chianti_1999,dere_chianti-atomic_2001,landi_chianti-atomic_2002,young_chianti-atomic_2003,landi_chianti-atomic_2006,landi_chianti-atomic_2006-1,dere_chianti_2009,landi_chiantiatomic_2009,young_chiantiatomic_2009,landi_chiantiatomic_2012,landi_chiantiatomic_2013,del_zanna_chianti_2015,young_chianti_2016} is an essential tool for analyzing and modeling spectra of optically-thin astrophysical plasmas such as the solar corona. It is primarily used in the study of the solar atmosphere though it has broader astrophysical applications as well \citep[see Figure 4 of][]{young_chianti_2016}. The database provides information on atomic transitions for all ions of over 30 different elements, from hydrogen to zinc. For a given ion, CHIANTI provides wavelengths and energies (among other information) for many thousands of atomic transitions as well as various derived quantities, including the ionization and recombination rates, energy level populations, and spectral line intensities. Additionally, CHIANTI provides multiple measurements of elemental abundances in both the corona and photosphere. Data and routines are also included for computing the free-free and free-bound continuum emission. 

The CHIANTI project is an international collaboration between the University of Cambridge, the University of Michigan, and George Mason University and is an invaluable asset to the solar physics community. Version 1.0 of the database was released in 1995 and at the time of writing, the current version is 8.0.7. Users typically interact with the database via the provided Interactive Data Language (IDL) routines or the more recently-released ChiantiPy package \citep{landi_chiantiatomic_2012,barnes_chiantipy_2017}, an interface to CHIANTI implemented in the Python programming language. All work presented in this thesis makes heavy use of the CHIANTI atomic database via the fiasco Python package (see \autoref{ap:fiasco}).

\section{Spectral Line Formation}\label{sec:line_formation}

The solar corona is \textit{optically thin}, meaning that all emitted photons are observed and that these photons are not absorbed or scattered between the emission site and detector. Because these photons travel uninterrupted, they provide a direct signature of the properties of the coronal plasma. One of the primary mechanisms for the formation of spectral emission lines in the solar corona is the spontaneous radiative decay of an electron in an excited state $j$ to a lower energy state $i$,
\begin{equation}\label{eq:radiative_decay}
    X_{k,j} \to X_{k,i} + h\nu_{ji},
\end{equation}
where $X_k$ is an ion of element $X$ in ionization stage $k$, $\nu_{ji}$ is the frequency of the atomic transition, and $h\nu_{ji}$ is the energy of the emitted photon.

The intensity of a spectral line for an atomic transition of wavelength $\lambda_{ji}=c/\nu_{ji}$, where $c$ is the speed of light in a vacuum, is given by,
\begin{equation}\label{eq:intensity}
    I(\lambda_{ji}) = \frac{1}{4\pi}\int_\textup{LOS}\dd{h}P(\lambda_{ji}),
\end{equation}
where the integration is taken along the line-of-sight (LOS) between the observer and the emission site and $P(\lambda_{ji})$ is the \textit{emissivity}, or the radiative power per unit volume. The emissivity is given by,
\begin{equation}\label{eq:emissivity}
    P(\lambda_{ji}) = \frac{hc}{\lambda_{ji}}n_{X,k,j}A_{ji},
\end{equation}
where $n_{X,k,j}$ is the number density of $X_k$ ions in excited state $j$ and $A_{ji}$ is the probability of spontaneous emission, often referred to as the Einstein coefficient \citep{bradshaw_collisional_2013,del_zanna_solar_2018}.

In general, it is quite difficult to determine $n_{X,k,j}$, the density of ions in excited state $j$. As such, we can rewrite $n_{X,k,j}$ as,
\begin{align}\label{eq:nj}
    n_{X,k,j} &= \frac{n_{X,k,j}}{n_{X,k}}\frac{n_{X,k}}{n_X}\frac{n_X}{n_H}\frac{n_H}{n_e}n_e, \nonumber\\
              &= N_{X,k,j}f_{X,k}\textup{Ab}(X)\frac{n_H}{n_e}n_e,
\end{align}
where $n_e$ is the electron density, $\textup{Ab}(X)=n_X/n_H$ is the abundance of element $X$ relative to hydrogen, $n_H/n_e$ is the ratio of hydrogen ions to electrons, often approximated as $n_H/n_e\approx0.83$, $N_{X,k,j}=n_{X,k,j}/n_{X,k}$ is the population of level $j$ or the fraction of $X_k$ ions in excited state $j$, and $f_{X,k}=n_{X,k}/n_X$ is the population fraction of ion $X_k$ \citep{del_zanna_solar_2018}. Plugging \autoref{eq:nj} into \autoref{eq:emissivity} yields a more convenient expression for the emissivity,
\begin{equation}\label{eq:emissivity_simplified}
    P(\lambda_{ji}) = 0.83\frac{hc}{\lambda_{ji}}\textup{Ab}(X)f_{X,k}N_{X,k,j}A_{ji}n_e.
\end{equation}
Both $\textup{Ab}(X)$ and $A_{ji}$, which is a function of electron temperature, $T_e$, can be looked up in the CHIANTI atomic database (see \autoref{sec:chianti}). $N_{X,k,j}$ is a function of both $T_e$ and $n_e$ and can be computed by assuming the the excitation and deexcitation processes are in equilibrium. This is discussed in \autoref{subsec:collisional_excitation} and \autoref{subsec:level_pops}. $f_{X,k}$ is primarily a function of $T_e$ and is discussed in \autoref{subsec:ionization_recombination} and \autoref{sec:ioneq}. Thus, for a given distribution of $T_e$ and $n_e$ along the LOS, we can compute the intensity of a transition $\lambda_{ji}$ using \autoref{eq:emissivity_simplified} and \autoref{eq:intensity}.

\subsection{Collisional Excitation of Atomic Levels}\label{subsec:collisional_excitation}

For a photon to be produced by spontaneous radiative decay from excited state $j$ to lower energy state $i$ (see \autoref{eq:radiative_decay}), the ion must first be excited into state $j$. In the solar atmosphere, the most important excitation process is the inelastic collisions between ions and free electrons,
\begin{equation}\label{eq:inelastic_collisions}
    X_{k,i} + e(E_{\textup{initial}}) \to X_{k,j} + e(E_{\textup{final}})
\end{equation}
where $e$ denotes the free electron and $E_{\textup{initial}}$ and $E_{\textup{final}}$ are the initial and final energies of the electron, respectively \citep{phillips_ultraviolet_2008}. The energies of levels $i$ and $j$ are $E_i$ and $E_j$, respectively. If $E_i<E_j$, $X_k$ has been \textit{collisionally excited} from a lower to a higher energy state and the free electron has lost an amount of energy equal to the separation between these two levels,
\begin{equation*}
    E_{\textup{final}} - E_{\textup{initial}}  = E_i - E_j.
\end{equation*}
Conversely, if $E_i>E_j$, $X_k$ is \textit{collisionally deexcited} from $i$ to $j$ and the free electron gains an amount of energy equal to $E_i - E_j$.

% Production of satellite lines via dielectronic recombination important for levels above the ionization potential

In order to understand how energy levels are populated and depopulated by collisions, it is necessary to compute the rate at which collisions occur in a plasma with electron temperature $T_e$ for an ion $X_k$. I will now derive the excitation and deexcitation rate coefficients. This derivation closely follows the treatment in sections 3.2 and 3.3 of \citet{del_zanna_solar_2018} as well as section 4.2 of \citet{phillips_ultraviolet_2008}.  

The rate coefficient for collisional excitation is given by,
\begin{equation}\label{eq:rate_coefficient}
    C^e_{ij} = \int_{v_0}^{\infty}\dd{v}v\sigma_{ij}(v)f(v),
\end{equation}
where $v$ is the electron velocity, $\sigma_{ij}(v)$ is the cross-section for inelastic collisions between the ion and electron, and $f(v)$ is the velocity distribution function of the electrons. Additionally, $v_0$ is the threshold velocity such that $m_ev_0^2/2 = E_j - E_i$, where $m_e$ is the mass of the electron. Any electron with $v<v_0$ will not be able to excite atom from level $i$ to $j$. It is commonly assumed that the distribution of free electrons in the solar atmosphere is in thermodynamic equilibrium such that it is well-described by a Maxwell-Boltzmann distribution\footnote{Observations of non-thermal particles \citep[e.g][]{dzifcakova_diagnostics_2011} suggest that the distribution of free electrons in the solar corona may be better described by a $\kappa$-distribution. For more details see \citet{cranmer_suprathermal_2014} or the comprehensive review by \citet{dudik_nonequilibrium_2017}.},
\begin{equation}\label{eq:maxwellian}
    f(v) = 4\pi v^2 \left(\frac{m_e}{2\pi k_BT_e}\right)^{3/2}\exp{\left(-\frac{m_ev^2}{2k_BT_e}\right)},
\end{equation}
where $k_B$ is the Boltzmann constant. Plugging \autoref{eq:maxwellian} into \autoref{eq:rate_coefficient},
\begin{equation*}
    C^e_{ij} =  4\pi\left(\frac{m_e}{2\pi k_BT_e}\right)^{3/2}\int_{v_0}^{\infty}\dd{v}v^3\sigma_{ij}(v)\exp{\left(-\frac{m_ev^2}{2k_BT_e}\right)},
\end{equation*}
and making the change of variables $E=m_ev^2/2$,
\begin{align}\label{eq:rate_coefficient_e}
    C^e_{ij} &= 4\pi\left(\frac{m_e}{2\pi k_BT_e}\right)^{3/2}\int_{E_0}^{\infty}\dd{E}\frac{2E}{m_e^2}\sigma_{ij}(E)\exp{\left(-\frac{E}{k_BT_e}\right)},\nonumber\\
    &= \sqrt{\frac{8}{\pi m_e}}(k_BT_e)^{-3/2}\int_{E_0}^{\infty}\dd{E}E\sigma_{ij}(E)\exp{\left(-\frac{E}{k_BT_e}\right)},\nonumber\\
    &= \sqrt{\frac{8k_BT_e}{\pi m_e}}\int_{E_0}^{\infty}\dd{\left(\frac{E}{k_BT_e}\right)}\frac{E}{k_BT_e}\sigma_{ij}(E)\exp{\left(-\frac{E}{k_BT_e}\right)},
\end{align}
where $E_0=mv_0^2/2=E_j-E_i$ is the minimum electron energy required to excite the ion from $i$ to $j$.

The cross-section for excitation by inelastic collisions can be expressed as,
\begin{equation}\label{eq:cross_section}
    \sigma_{ij}(E) = \pi a_0^2 \Omega_{ij}(E)\frac{I_H}{\omega_iE},
\end{equation}
where $a_0$ is the Bohr radius, $I_H$ is the ionization potential of hydrogen, $\omega_i$ is the statistical weight of level $i$, and $\Omega_{ij}$ is the dimensionless collision strength. It should be noted that $\Omega_{ij}$ is symmetric such that $\Omega_{ij}(E)=\Omega_{ji}(E^{\prime})$, where $E^{\prime}=E - E_{ij} = E - (E_j - E_i)$ is the final energy of the electron after it has been scattered. Plugging \autoref{eq:cross_section} into \autoref{eq:rate_coefficient_e},
\begin{equation}
    C^e_{ij} = I_Ha_0^2\sqrt{\frac{8\pi}{k_Bm_e}}\omega_i^{-1}T_e^{-1/2}\int_{E_0}^{\infty}\dd{\left(\frac{E}{k_BT_e}\right)}\Omega_{ij}(E)\exp{\left(-\frac{E}{k_BT_e}\right)}.
\end{equation}
Exploiting the symmetry of $\Omega_{ij}$ and making a change of variables to $E^{\prime}$ gives,
\begin{align}\label{eq:rate_coefficient_final}
    C^e_{ij} &= I_Ha_0^2\sqrt{\frac{8\pi}{k_Bm_e}}\omega_i^{-1}T_e^{-1/2}\int_{E_0}^{\infty}\dd{\left(\frac{E}{k_BT_e}\right)}\Omega_{ji}(E^\prime)\exp{\left(-\frac{E}{k_BT_e}\right)} \nonumber\\
    &= I_Ha_0^2\sqrt{\frac{8\pi}{k_Bm_e}}\omega_i^{-1}T_e^{-1/2}\int_{0}^{\infty}\dd{\left(\frac{E^\prime}{k_BT_e}\right)}\Omega_{ji}(E^\prime)\exp{\left(-\frac{E^\prime + E_{ij}}{k_BT_e}\right)} \nonumber\\
    &= I_Ha_0^2\sqrt{\frac{8\pi}{k_Bm_e}}T_e^{-1/2}\frac{\Upsilon_{ij}}{\omega_i}\exp{\left(-\frac{E_{ij}}{k_BT_e}\right)}
\end{align}
where the term $\Upsilon_{ij}$, originally introduced by \citet{seaton_electron_1953}, is called the effective collision strength (or alternatively the Maxwellian-averaged collision strength) and is defined as,
\begin{equation}\label{eq:effective_collision_strength}
    \Upsilon_{ij} = \int_{0}^{\infty}\dd{\left(\frac{E}{k_BT_e}\right)}\Omega_{ji}(E)\exp{\left(-\frac{E}{k_BT_e}\right)},
\end{equation}
where $E$ is now the final energy of the electron after the collision.

In general, computing cross-sections for excitations by collisions with free-electrons is very difficult and time consuming and requires the use of sophisticated atomic codes which properly account for the energy levels of the target ion and the detailed physics of the interaction between the free electron and target ion \citep[][section 4.2.3]{phillips_ultraviolet_2008,bautista_theoretical_2000}. \citet{burgess_analysis_1992} computed fit coefficients to $\Upsilon$ as a function of $T_e$ in terms of compact, dimensionless variables for a large number of atomic transitions. Reduced fit parameters for $\Upsilon$ are provided in the CHIANTI atomic database using the methods of \citet{burgess_analysis_1992} for all relevant transitions such that effective collision strengths can be efficiently computed for arbitrary $T_e$. \autoref{fig:upsilon} shows $\Upsilon$ as a function of $T_e$ for a number of transitions of Fe XII.

% spell-checker: disable %
\begin{pycode}[chapter3]
fig = plt.figure(figsize=texfigure.figsize(pytex,scale=1,))
ax = fig.gca()
T = np.logspace(5.5,7.5,100) * u.K
ion = fiasco.Ion('Fe 12', T)
ups = ion.effective_collision_strength()
index = np.where(np.logical_and(ups[0,:] < 1,ups[-1,:]>1e-3))[0][:100]
ax.plot(T, ups[:, index], color=PALETTE[0], alpha=0.25)
ax.set_xlabel(r'$T_e$ $[\si{\kelvin}]$')
ax.set_ylabel(r'$\Upsilon$')
ax.set_xscale('log')
ax.set_yscale('log')
ax.set_xlim(T[[0,-1]].value)
ax.set_ylim(1e-3,1)
tfig = ch3.save_figure('upsilon', fext='.pgf')
tfig.caption = r'Effective collision strength, $\Upsilon$, as a function of $T_e$ for 100 selected transitions in Fe XII. $\Upsilon$ was interpolated to $T_e$ using fit coefficients provided by the CHIANTI atomic database and computed using the method of \citet{burgess_analysis_1992}'
\end{pycode}
\py[chapter3]|tfig|
% spell-checker: enable %

The rate coefficient for deexcitation can also be computed using \autoref{eq:rate_coefficient_final}, the excitation rate coefficient. Under the assumption of thermodynamic equilibrium, the processes of excitation and deexcitation by collisions must be in balance such that 
\begin{equation}\label{eq:eqbm_balance}
    n_in_eC_{ij}^e = n_jn_eC_{ji}^d,
\end{equation}
where $C_{ji}^d$ is the rate coefficient for collisional deexcitation, and the populations of the two levels are in Boltzmann equilibrium,
\begin{equation}\label{eq:boltzmann_eqbm}
    \frac{n_i}{n_j} = \frac{\omega_i}{\omega_j}\exp{\left(\frac{E_{ij}}{k_BT_e}\right)},
\end{equation}
where $\omega_i$ and $\omega_j$ are the statistical weights of levels $i$ and $j$, respectively. Combining \autoref{eq:eqbm_balance} and \autoref{eq:boltzmann_eqbm} gives an expression for $C_{ji}^d$, the deexcitation rate coefficient,
\begin{align}\label{eq:dex_rate_coefficient}
    C_{ji}^d &= \frac{\omega_i}{\omega_j}C_{ij}^e\exp{\left(\frac{E_{ij}}{k_BT_e}\right)},\nonumber\\
    &= I_Ha_0^2\sqrt{\frac{8\pi}{k_Bm_e}}T_e^{-1/2}\frac{\Upsilon_{ij}}{\omega_j}.
\end{align}

\subsection{Level Populations}\label{subsec:level_pops}

In optically-thin, astrophysical plasmas, it is often assumed that the processes which influence populations of atomic energy levels are decoupled from those processes which influence the charge state of the atom (see \autoref{subsec:ionization_recombination}). This is because changes in the energy level populations occur much more frequently than changes in the charge state. Another common approximation is that energy levels are populated primarily by collisional excitation and depopulated by spontaneous radiative decay and that these processes occur primarily between the ground state $g$ and an excited state $j$. Assuming a steady-state equilibrium between these processes gives,
\begin{equation}\label{eq:coronal_model}
    n_{X,k,g}n_eC_{gj}^e = n_{X,k,j}A_{jg},
\end{equation}
where the left-hand side corresponds to processes that populate $j$ and the right-hand side corresponds to processes that depopulate $j$. Taken together, these assumptions are often referred to as the \textit{coronal model approximation} \citep{bradshaw_collisional_2013,del_zanna_solar_2018}. 

The coronal model approximation assumes a two-level system ($g$ and $j$) in which the only two competing processes are excitation by collisions and spontaneous radiative decay. However, an ion may have so-called \textit{metastable levels} where the probability of spontaneous radiative decay is very low such that depopulation by collisional deexcitation is not negligible \citep{phillips_ultraviolet_2008,del_zanna_solar_2018}. In this case, the level population calculation must account for transitions between excited states such that the two-level approximation is no longer appropriate. Thus, for a multi-level atom, the system of equations required to calculate the population of $n_j$ is (temporarily dropping the $X,k$ subscripts),
\begin{equation}\label{eq:level_pop}
\sum_{i>j}n_iA_{ij} + n_e\sum_{i>j}n_iC_{ij}^d + n_e\sum_{i<j}n_iC_{ij}^e
= n_j\left(\sum_{i<j}A_{ji} + n_e\sum_{i<j}C_{ji}^d + n_e\sum_{i>j}C_{ji}^e\right),
\end{equation}
where the left-hand side denotes processes which populate level $j$ and the right-hand side denotes processes that depopulate level $j$ \citep{del_zanna_solar_2018}. $C_{ij}^e$ and $C_{ji}^d$ can be computed from \autoref{eq:rate_coefficient_final} and \autoref{eq:dex_rate_coefficient}, respectively. Values for $A_{ji}$ as a function of $T_e$ are available in CHIANTI. Additional processes such as collisional excitation by protons or photoionization by an external radiation field may also influence $n_j$ \citep[see sections 3.4.1 and 3.4.2 of][]{del_zanna_solar_2018}.

In general, the level population $n_j$ is a function of both temperature and density. As \autoref{eq:level_pop} is a system of $J$ coupled equations, where $J$ is the total number of energy levels of the ion, calculating $n_j$ requires solving a $J\times J$ matrix equation and can be very computationally expensive, depending on the number of energy levels and relevant atomic transitions. \autoref{fig:level-pop} shows the level populations of several energy levels of O II as a function of electron energy at $10^6$ K. The resulting relative level population for level $j$ of charge state $k$ of element $X$, $N_{X,k,j}$, can then be used to compute the emissivity for transition $\lambda_{ji}$ (\autoref{eq:emissivity_simplified}) and the resulting spectral line intensity (\autoref{eq:intensity}).

% spell-checker: disable %
\begin{pycode}[chapter3]
ion = fiasco.Ion('O 2', [1e6]*u.K)
n = np.logspace(3,12,100)*u.cm**(-3)
pop = ion.level_populations(n)

# Function for making spectroscopic notation for energy levels
def make_level_label(i):
    spec = r'$^{s}\mathrm{{{l}}}_{{{j}}}$'.format(
        j = str(Fraction(ion[i].total_angular_momentum.value)),
        s = ion._elvlc['multiplicity'][i],
        l = ion[i].orbital_angular_momentum_label
    )
    return f"{ion[i].level}: {' '.join(ion[i].configuration.split('.'))} {spec}"

fig = plt.figure(figsize=texfigure.figsize(pytex,scale=1,))
ax = fig.gca()
num_levels = 5
for i in range(num_levels):
    ax.plot(n, pop[0,:,i], label=make_level_label(i), color=PALETTE[i])
ax.set_xscale('log')
ax.set_yscale('log')
ax.set_ylim(1e-3,2)
ax.set_xlim(n[0].value,n[-1].value)
ax.set_xlabel(r'$n_e$ $[\si{\cm}^{-3}]$')
ax.set_ylabel(r'$N_j$')
ax.xaxis.set_major_locator(matplotlib.ticker.LogLocator(numticks=5))
ax.legend(loc=4,frameon=False,ncol=2)
tfig = ch3.save_figure('level-pop', fext='.pgf')
tfig.caption = r'Level population of the first five levels of O II as a function of electron density, $n_e$, at $T_e=10^6$ K. Note that the ground state is the most abundant for all $n_e$. The level population is normalized to the total number of O II ions such that $\sum_jN_j=1$. Adapted from Fig. 4.3 of \citet{phillips_ultraviolet_2008}.'
\end{pycode}
\py[chapter3]|tfig|
% spell-checker: enable %

\subsection{Processes which Affect the Ion Charge State}\label{subsec:ionization_recombination}

In addition to the relative populations of each energy level of the ion, one must also know the population of each \textit{charge state} of the ion in order to compute the emissivity (\autoref{eq:emissivity_simplified}). The relative population fraction of a charge state $k$ of element $X$, denoted $f_{X,k}=n_{X,k}/n_X$, is the number of ions of element $X$ in charge state $k$ relative to the total number of ions of element $X$. Ion charge states are determined by two primary processes: \textit{ionization}, in which a bound electron is freed by some external perturbation, and \textit{recombination}, in which a a free electron is captured by the ion. Thus, the time evolution of the population fraction $f_k$ (temporarily dropping the element label) is given by,
\begin{equation}\label{eq:population_fraction}
    \frac{d}{dt}f_k = n_e(\alpha_{k-1}^I f_{k-1} + \alpha_{k+1}^R f_{k+1} - \alpha_{k}^I f_k - \alpha_k^R f_k),
\end{equation}
where $\alpha_k^I$ and $\alpha_k^R$ are the ionization and recombination rates of charge state $k$, respectively, and the population fractions are subject to the constraint $\sum_kf_k=1$ \citep{del_zanna_solar_2018}. In general, the derivative on the left-hand side is the comoving derivative such that $\frac{d}{dt}=\frac{\partial}{\partial t} + v\frac{\partial}{\partial s}$. Note that ionization from lower charge states and recombination from higher charge states are source terms while ionization and recombination out of the current charge state are sinks. Solutions to \autoref{eq:population_fraction} are discussed in \autoref{sec:ioneq}.

\subsubsection{Ionization}

In optically-thin astrophysical plasmas such as the solar corona, the dominant processes contributing to the total ionization rate are \textit{collisional ionization} and \textit{excitation-autoionization} \citep{bradshaw_collisional_2013}. Thus, the total ionization rate can be written as,
\begin{equation}\label{eq:total_ionization_rate}
    \alpha^I = \alpha^\textup{CI} + \alpha^\textup{EA},
\end{equation}
where $\alpha^\textup{CI}$ and $\alpha^\textup{EA}$ are the ionization rates due collisional ionization and excitation-autoionization, respectively. 

In the process of collisional ionization, a free electron collides with an ion $X_k$ and frees a bound electron. Following the notation of \citet{bradshaw_collisional_2013,mason_spectroscopic_1994}, this can be expressed as,
\begin{equation}\label{eq:collisional_ionization}
    X_{k,i} + e \to X_{k+1,i^\prime} + 2e,
\end{equation}
where $i^\prime$ denotes the final energetic state of $X_{k+1}$. $\alpha_{CI}$ can be computed in a similar manner to $C^e_{ij}$ by integrating the velocity-weighted collision cross-section over a Maxwell-Boltzmann distribution. Using the result from \autoref{eq:rate_coefficient_e}, the ionization rate due to collisional ionization can be written as,
\begin{equation}
    \alpha^\textup{CI} = \sqrt{\frac{8}{\pi m_e}}(k_BT_e)^{-3/2}\int_I^{\infty}\dd{E}E\sigma_{CI}(E)\exp{\left(-\frac{E}{k_BT_e}\right)}
\end{equation}
where $E$ is the energy of incident electron, $\sigma_{CI}$ is the collisional ionization cross-section and $I$ is the ionization energy of the initially-bound electron \citep{del_zanna_solar_2018}. Making the change of variables $x=(E-I)/k_BT_e$ gives,
\begin{equation}\label{eq:collisional_ionization_x}
\begin{aligned}
    \alpha^\textup{CI} = \sqrt{\frac{8k_BT_e}{\pi m_e}}\exp{\left(-\frac{I}{k_BT_e}\right)}&\left(\int_0^{\infty}\dd{x}x\sigma_{CI}(k_BT_ex+I)e^{-x} \right. \\
    &\left. + \frac{I}{k_BT_e}\int_0^{\infty}\dd{x}\sigma_{CI}(k_BT_ex+I)e^{-x}\right).
\end{aligned}
\end{equation}
Notice that both integrals have the same form and can be evaluated using Gauss-Laguerre quadrature,
\begin{equation}
    \int_0^\infty\dd{x}f(x)e^{-x} \approx \sum_{i=1}^n w_if(x_i),
\end{equation}
where $x_i$ is the zero of the $i$-th Laguerre polynomial and $w_i$ are the associated weights \citep[see Equation 25.4.45 of][]{abramowitz_handbook_1972}. Note that in both terms, evaluating $f(x_i)$ requires evaluating the collisional ionization cross-section, $\sigma_{CI}$.

As in the case of collisional excitation cross-section, evaluating $\sigma_\textup{CI}$ is non-trivial. For ions in the hydrogen and helium isoelectronic sequences (i.e. ions with the same number of electrons as hydrogen or helium), $\sigma_\textup{CI}$ can be calculated using the fitting formula of \citet{fontes_fully_1999}. Additionally, \citet{dere_ionization_2007} provide fit coefficients for collisional ionization cross-sections for a large number of ions using the method of \citet{burgess_analysis_1992}. Fit parameters for both of these methods are provided by the CHIANTI database and can be used to efficiently compute $\sigma_\textup{CI}$ as a function of $T_e$.

In the case of excitation-autoionization, if an ion is collisionally excited to a level above the ionization threshold, it can autoionize, resulting in a free electron and an ion in a higher charge state, but lower energy state. This process can be written as,
\begin{equation}\label{eq:excitation_autoionization}
    X_{k,i^\prime} + e(E_1) \to X_{k,j} + e(E_2) \to X_{k+1,i} + e(E_2) + e^\prime(E),
\end{equation}
where $e^\prime$ is the recently freed electron \citep{phillips_ultraviolet_2008}. In the first step, $X_k$ is excited from $i^\prime$ to $j$ and the in the second step, $X_k$ decays from $j$ to $i$ and emits an electron $e^\prime$, producing a higher charge state $k+1$. This is only possible provided $E_1 - E_2$ is greater than the ionization threshold of $X_k$ \citep{bradshaw_collisional_2013}. The ionization rate due to excitation-autoionization, $\alpha^\textup{EA}$, can be computed using an expression analogous to \autoref{eq:rate_coefficient_final}, but replacing $\Upsilon$ with the appropriate effective collision strengths for excitation-autoionization. Scaled fit parameters to these collision strengths, as computed by \citet{dere_ionization_2007} using the method of \citet{burgess_analysis_1992}, are provided in the CHIANTI database. \autoref{fig:ionization-recombination-rates} shows the total (solid blue), collisional (dashed blue), and excitation-autoionization (dot-dashed blue) ionization rates as a function of temperature for Fe XVI.   

% spell-checker: disable %
\begin{pycode}[chapter3]
ion = fiasco.Ion('Fe 16', np.logspace(4,9,100)*u.K)
fig = plt.figure(figsize=texfigure.figsize(pytex,scale=1,))
ax = fig.gca()
# ionization
ax.plot(ion.temperature,ion.ionization_rate(),color=PALETTE[0],ls='-',label=r'$\alpha^I$')
ax.plot(ion.temperature,ion.direct_ionization_rate(),color=PALETTE[0],ls='--',label=r'$\alpha^\textup{CI}$')
ax.plot(ion.temperature,ion.excitation_autoionization_rate(),color=PALETTE[0],ls='-.',label=r'$\alpha^\textup{EA}$')
# recombination
ax.plot(ion.temperature,ion.recombination_rate(),color=PALETTE[1],ls='-',label=r'$\alpha^R$')
ax.plot(ion.temperature,ion.radiative_recombination_rate(),color=PALETTE[1],ls='--',label=r'$\alpha^\textup{RR}$')
ax.plot(ion.temperature,ion.dielectronic_recombination_rate(),color=PALETTE[1],ls='-.',label=r'$\alpha^\textup{DR}$')
ax.set_xscale('log')
ax.set_yscale('log')
ax.set_ylim(1e-12,1e-9)
ax.set_xlim(ion.temperature[[0,-1]].value)
ax.set_xlabel(r'$T_e$ $[\si{\kelvin}]$')
ax.set_ylabel(r'$\alpha$ $[\si{\cm}^3\,\si{\second}^{-1}]$')
ax.legend(loc=4,ncol=1,frameon=False)
tfig = ch3.save_figure('ionization-recombination-rates', fext='.pgf')
tfig.caption = r'Ionization (blue) and recombination (orange) rates as a function of electron temperature, $T_e$, for Fe XVI. The constituent rates are denoted by dashed and dot-dashed lines. Note that the recombination rate dominates at low $T_e$ while the ionization rate dominates at high $T_e$, as expected.'
\end{pycode}
\py[chapter3]|tfig|
% spell-checker: enable %

\subsubsection{Recombination}

Along with ionization, recombination, the capture of a free electron by the target ion, is the other primary process which determines the charge state of the ion. In an optically-thin plasma, the two dominant processes that contribute to the total recombination rate are \textit{radiative recombination} and \textit{dielectronic recombination} \citep{bradshaw_collisional_2013}. As in \autoref{eq:total_ionization_rate}, the total recombination rate can be written as,
\begin{equation}\label{eq:total_recombination_rate}
    \alpha^\textup{R} = \alpha^\textup{RR} + \alpha^\textup{DR},
\end{equation}
where $\alpha^\textup{RR}$ and $\alpha^\textup{DR}$ are the recombination rates due to radiative and dielectronic recombination, respectively.

In the case of radiative recombination, a free electron is captured into a bound state and a photon is emitted. This process can be expressed as,
\begin{equation}\label{eq:radiative_recombination}
    X_{k+1,j} + e(E) \to X_{k,i} + h\nu_{ji}.
\end{equation}
Note that energy conservation requires that $h\nu_{ji}=E_j - E_i + E$ such that the energy of the emitted photon must be equal to that of the initial kinetic energy of the electron plus the energy differential between levels $j$ and $i$. In general, computing $\alpha^\textup{RR}$ is nontrivial as it requires calculating the photoionization cross-section between $i$ and $j$ as well as the level population of excited state $j$ \citep[see Equation 4.47 of][]{phillips_ultraviolet_2008}. \citet{shull_ionization_1982} calculated $\alpha^\textup{RR}$ for C, N, I, Ne, Mg, Si, S, Ar, Ca, Fe, and Ni using a relatively simple fit function,
\begin{equation}\label{eq:shull_radiative_recombination_rate}
    \alpha^\textup{RR}(T_e) = A\left(\frac{T_e}{10^4}\right)^{-\eta},
\end{equation}
where $A$ and $\eta$ are determined by fitting \autoref{eq:shull_radiative_recombination_rate} to tabulated values of $\alpha^\textup{RR}$ from the literature. \citet{badnell_radiative_2006} later improved on this result by computing distorted-wave photoionization cross-sections for all ions up to and including Zn and calculating fit parameters from these results for a more sophisticated analytical fitting function for $\alpha^\textup{RR}$ \citep[see Equation 4 of][]{verner_atomic_1996}. The CHIANTI database uses the results of both \citet{shull_ionization_1982} and \citet{badnell_radiative_2006}, as appropriate, to efficiently compute $\alpha^\textup{RR}$ as a function of electron temperature.

Dielectronic recombination is the inverse process of excitation-autoionization (\autoref{eq:excitation_autoionization}) and can be expressed as,
\begin{equation}\label{eq:dielectronic_recombination}
    X_{k+1,i^\prime} + e \to X_{k,j} \to X_{k,i} + h\nu_{ji}.
\end{equation}
Note that in the intermediate step of this process, $X_{k,j}$ is in a doubly-excited state wherein a free electron has been captured and an already-bound electron has been excited to a higher energy level $j$. When the ion decays from $j$ to $i$ and a photon is emitted, the captured electron $e$ remains in an excited energy level \citep{bradshaw_collisional_2013}. The recombination rate due to dielectronic recombination, $\alpha^\textup{DR}$, can be evaluated in a similar manner to $\alpha^\textup{RR}$. \citet{shull_ionization_1982} fit an analytical form for $\alpha^\textup{DR}(T_e)$ to data from the literature and provide fit parameters for the same set of ions as $\alpha^\textup{RR}$. Improved analytical forms and fit parameters for $\alpha^\textup{DR}$ to results obtained from a distorted-wave approximation calculation are provided in a series of 14 papers by \citet{badnell_dielectronic_2003}. As in the case of radiative recombination, the approaches of both \citet{shull_ionization_1982} and \citet{badnell_dielectronic_2003} are used to compute $\alpha^\textup{DR}(T_e)$ in CHIANTI. \autoref{fig:ionization-recombination-rates} shows the total (solid orange), radiative (dashed orange), and dielectronic (dot-dashed orange) recombination rates as a function of $T_e$ for Fe XVI. Note that dielectronic recombination dominates at high temperatures while radiative recombination is more important at low temperatures ($T_e\lesssim3\times10^4$ K).

\subsection{The Charge State in Equilibrium}\label{sec:ioneq}

As noted in \autoref{subsec:ionization_recombination}, there are many methods for computing the ionization and recombination rate coefficients, primarily as a function of electron temperature, $T_e$. One can then use these rate coefficients to compute the relative population fraction of all ions of particular element as a function of temperature. When computing these population fractions for a high-temperature, low-density plasma such as the solar corona, it is often assumed that the charge state of the plasma is in \textit{ionization equilibrium} or that the processes which populate and depopulate a particular charge state are in balance. Under this assumption, \autoref{eq:population_fraction} becomes (temporarily dropping the element label $X$),
\begin{equation}\label{eq:ionization_equilibrium}
    n_e(\alpha_{k-1}^I f_{k-1} + \alpha_{k+1}^R f_{k+1} - \alpha_{k}^I f_k - \alpha_k^R f_k) = 0.
\end{equation}

For element $X$ with $Z+1$ total charge states, where $Z$ is the atomic number of $X$, \autoref{eq:ionization_equilibrium} represents a system of $Z+1$ linear, homogeneous equations and can be expressed in matrix form as,
\begin{equation}\label{eq:ionization_equilibrium_matrix}
    \mathbf{A}\mathbf{F} = \mathbf{0},
\end{equation}
where $\mathbf{F}=(f_1,f_2,\ldots,f_k,\ldots,f_{Z+1})$ is a column vector of length $Z+1$, $\mathbf{A}$ is a $Z+1{\times}Z+1$ matrix containing all of the ionization and recombination rates, and $\mathbf{0}$ is the zero vector with $Z+1$ entries. For a given element $X$ and electron temperature $T_e$, one can compute $\alpha^\textup{I}$ and $\alpha^\textup{R}$ (e.g. using CHIANTI) for all $k$ to find $\mathbf{A}$. To solve \autoref{eq:ionization_equilibrium_matrix}, consider the singular value decomposition (SVD) of $\mathbf{A}$,
\begin{equation}
    \mathbf{A} = \mathbf{U}\mathbf{W}\mathbf{V}^T,
\end{equation}
where $\mathbf{W}$ is a $Z+1{\times}Z+1$ diagonal matrix with singular values ($w_k$) along the diagonal and $\mathbf{U}$ and $\mathbf{V}$ are  square $Z+1{\times}Z+1$ matrices whose columns are orthonormal. Provided $\mathbf{A}$ is singular (i.e. $\det{(\mathbf{A})}=0$), any column of $\mathbf{V}$ for which the corresponding $w_k=0$ is a solution of \autoref{eq:ionization_equilibrium_matrix} \citep{press_numerical_1992}. 

\autoref{fig:ionization-equilibrium} shows the population fractions, $f_{X,k}$, for every ionization state of Fe as a function of electron temperature, assuming ionization equilibrium. As the electron temperature rises, increasingly higher charge states are populated because the free electrons are more energetic and are capable of releasing more tightly-bound electrons. Conversely, at lower $T_e$, the free electrons have lower energy and recombine into the outer bound states of the target ions such that the lower charge states become more populated \citep{bradshaw_collisional_2013}. Note that the temperature at which each population fraction peaks is the point at which the ionization and recombination rates for that charge state are equal (see \autoref{eq:population_fraction}).

% spell-checker: disable %
\begin{pycode}[chapter3]
fe = fiasco.Element('Fe',np.logspace(4,9,1000)*u.K)
ioneq = fe.equilibrium_ionization()
fig = plt.figure(figsize=texfigure.figsize(pytex,scale=1,height_ratio=0.5))
ax = fig.gca()
for i in fe:
    ax.plot(fe.temperature, ioneq[:,i.charge_state],
            color=PALETTE[i.charge_state],)
    z = i.ionization_stage
    if (2 <= z <= 10) or (16 <= z <= 18) or (24 <= z <= 26):
        ax.text(fe.temperature[np.argmax(ioneq[:,i.charge_state])].value,
                np.max(ioneq[:,i.charge_state]).value,
                f'{fe.atomic_symbol} {roman.toRoman(i.ionization_stage)}',
                horizontalalignment='center', rotation=0,
                fontsize=8)
ax.set_xscale('log')
ax.set_xlim(fe.temperature[[0,-1]].value)
ax.set_ylim(0,1.01)
ax.yaxis.set_major_locator(matplotlib.ticker.FixedLocator(np.arange(0.2,1.2,0.2)))
ax.set_xlabel(r'$T_e$ $[\si{\kelvin}]$')
ax.set_ylabel(r'$f_{X,k}$')
tfig = ch3.save_figure('ionization-equilibrium', fext='.pgf')
tfig.caption = r'Ion population fractions for every ionization state of Fe as a function of $T_e$. The population fractions were computed assuming ionization equilibrium using \autoref{eq:ionization_equilibrium_matrix}. Note that increasingly higher ionization states become populated with increasing electron temperature and vice versa.'
\end{pycode}
\py[chapter3]|tfig|
% spell-checker: enable %

\subsection{Nonequilibrium Ionization}\label{sec:nei}

The population fraction, $f_k$, as a function of $T_e$ can be accurately determined by solving \autoref{eq:ionization_equilibrium} provided that the charge states are in equilibrium with the electron temperature of the plasma. This is a reasonable approximation provided that the electron temperature of the plasma changes sufficiently slowly as some finite amount time is required for the charge states to rearrange themselves following a dramatic change in temperature \citep{bradshaw_collisional_2013}. The time required for a charge state $k$ to reach equilibrium following a change in temperature from $T_0$ to $T_1$ in a plasma of density $n_e$ can be expressed as,
\begin{equation}\label{eq:cie_timescale}
    \tau_\textup{CIE} = \frac{1}{n_e(\alpha_{k-1}^I(T_1) f_{k-1}(T_0) + \alpha_{k+1}^R(T_1) f_{k+1}(T_0) - \alpha_{k}^I(T_1) f_k(T_0) - \alpha_k^R(T_1) f_k(T_0))},
\end{equation}
where $\tau_\textup{CIE}$ is often called the collisional ionization equilibration timescale (CIE). If $T_e$ changes from $T_0$ to $T_1$ in a time less than $\tau_\textup{CIE}$, then the charge state $k$ is out of equilibrium and $f_k$ must be computed by solving the full time-dependent population fraction equation (\autoref{eq:population_fraction}). \citet{smith_ionization_2010} computed the maximum $\tau_\textup{CIE}$ for several different elements and found that for a plasma of density $n_e=10^9$ cm$^{-3}$, the time needed to equilibrate to within 10\% of the equilibrium population fractions of Fe was at least $\ge200$ s for $T_e\ge10^6$ K and approached several thousand seconds for much higher temperatures ($10^9$ K).

In a low-density, high-temperature plasma such as the solar corona which can undergo temperature changes on timescales of a few hundred or even tens of seconds (e.g. flares, heating by nanoflares), the temperature implied by the equilibrium charge states may not be representative of the actual electron temperature. As such, correctly accounting for nonequilibrium ionization (NEI) is critical for understanding the radiative losses from the coronal plasma. \citet{macneice_numerical_1984} solved the full time-dependent ionization/recombination equations, including effects due to radiative and collisional excitation and deexcitation, in order to accurately model the spectrum of Ca in a flaring loop heated by an electron beam. Later, \citet{bradshaw_numerical_2009} developed a robust, explicit numerical scheme for solving \autoref{eq:population_fraction} and provided an exhaustive set of test cases for various temperature and density gradients. Many workers \citep[e.g.][]{hansteen_new_1993,reale_nonequilibrium_2008,bradshaw_radiative_2003,bradshaw_explosive_2006,bradshaw_what_2011,bradshaw_quantifying_2019} have also found that NEI is a critically important when attempting to accurately predict spectral intensities of an impulsively heated, low-density plasma. In particular, effects due to nonequilibrium ionization are likely to limit the observability of ``very hot'' plasma, one of the primary observable signatures of nanoflare heating (see \autoref{ch:inferring_hot_plasma}).

% spell-checker: disable %
\begin{pycode}[chapter3]
# Setup temperature and density profiles
t = np.arange(0,100,0.05) * u.s
thalf = t[-1]/2
Tmax = 1e7*u.K
Tmin = 1e5*u.K
T = np.where(t <= thalf, Tmin + t*(Tmax-Tmin)/(thalf),
             Tmax + (t - thalf)*(Tmin - Tmax)/(t[-1] - thalf))*Tmax.unit
n = 1e9*u.cm**(-3) * np.ones(t.shape)
# Compute NEI
el = synthesizAR.atomic.Element('iron', np.logspace(4,9,1000)*u.K)
nei = el.non_equilibrium_ionization(t,T,n)
# Interpolate IEQ populations
ioneq = el.equilibrium_ionization()
ioneq_interp = interp1d(el.temperature.value, ioneq.value.T, 
                        kind='cubic',fill_value='extrapolate')(T.value).T
# Plot
fig = plt.figure(figsize=texfigure.figsize(pytex,scale=1,height_ratio=0.55))
## Population fractions
ax = fig.gca()
for i in range(9,15):
    ax.plot(t, ioneq_interp[:,i], color=PALETTE[i],ls='--')
    ax.plot(t, nei[:,i], color=PALETTE[i],ls='-',
            label=f'{el.atomic_symbol} {roman.toRoman(el[i].ionization_stage)}')
ax.set_ylim(1e-6,1)
ax.set_yscale('log')
ax.set_xlim(t[[0,-1]].value)
ax.set_xlabel(r'$t$ $[\si{\second}]$')
ax.set_ylabel(r'$f_{X,k}$')
ax.yaxis.set_major_locator(matplotlib.ticker.LogLocator(numticks=4))
ax.tick_params(axis='x', which='both', pad=6)
## Temperature
ax2 = ax.twinx()
ax2.plot(t,T,color='k',ls='-',alpha=0.5)
ax2.set_xlim(t[[0,-1]].value)
ax2.set_ylim(Tmin.value,Tmax.value)
ax2.set_yscale('log')
ax2.set_ylabel(r'$T_e$ $[\si{\kelvin}]$')
ax2.tick_params(axis='y', which='both', pad=6)
ax.legend(ncol=3,bbox_to_anchor=(0.5,1),frameon=False,loc='lower center')
tfig = ch3.save_figure('nonequilibrium-ionization', fext='.pgf')
tfig.caption = r'Equilibrium (dashed) and nonequilibrium (solid) population fractions as a function of time, $t$, for Fe X through Fe XV. The time-dependent temperature profile, $T_e$, is shown on the right axis in black. The density is held constant at $n_e=10^9$ $\si{\cm}^{-3}$ for the entire simulation interval.'
\end{pycode}
\py[chapter3]|tfig|
% spell-checker: enable %

\autoref{fig:nonequilibrium-ionization} shows the population fractions of Fe X through Fe XV in equilibrium (dashed) and nonequilibrium (solid). In this simple example, the electron temperature (black) increases linearly from $10^5$ K to $10^7$ K over 50 s and then decreases linearly back to $10^5$ K over 50 s. The density is held constant at $10^9$ cm$^{-3}$ for the entire 100 s. Note that for all 6 ions shown, the population fractions are out of equilibrium for the entire 100 s and may differ by many orders of magnitude. In particular, the peaks of the nonequilibrium population fractions lag those of the equilibrium populations as a finite amount of time is required for the charge state to form following a change in $T_e$. Additionally, unlike the equilibrium populations, which track the electron temperature exactly, the nonequilibrium populations are not symmetric about the peak in $T_e$. The equilibrium populations were determined by computing the SVD of the matrix $\mathbf{A}$ in \autoref{eq:ionization_equilibrium_matrix} and the nonequilibrium populations were computed by solving \autoref{eq:population_fraction} using an implicit method (see \autoref{ap:nonequilibrium_implicit}).

\section{Continuum Emission}\label{sec:continuum}

\subsection{Free-free Emission}

While the coronal EUV spectrum is dominated by spectral line emission, continuum emission becomes important for wavelengths in the X-ray band (\SIrange{1}{100}{\angstrom}). There are two main types of continuum emission: free-free and and free-bound. Free-free emission, also known as \textit{bremsstrahlung} (or ``braking radiation''), is produced when an ion interacts with a free electron, reduces the momentum of the free electron, and, by conservation of energy and momentum, produces a photon. This process can be expressed as,
\begin{equation}\label{eq:free_free_process}
    X_k + e(E_1) \to X_k + e(E_2) + h\nu,
\end{equation}
where $E_1$ and $E_2$ are the initial and final energies of the electron \citep{del_zanna_solar_2018}. Bremsstrahlung is the primary mechanism for producing hard X-ray emission in hot (\SI{\ge e7}{\kelvin}) flare plasma. The emission per unit time, per unit volume, and per unit wavelength produced by the free-free process for an electron with a velocity $v$ in \autoref{eq:free_free_process} is given by,
\begin{equation}\label{eq:bremsstrahlung_single}
    P_{ff}(\lambda,v) = \frac{16\pi e^6Z^2}{3^{3/2}c^2m_e^2v\lambda^2}n_en_ig_{ff}(\lambda,v),
\end{equation}
where $n_i$ is the number density of the ions and $g_{ff}$ is the so-called free-free Gaunt factor which is a correction factor for the integral over the interaction cross-section \citep{rybicki_radiative_1979}.

Because the coronal plasma is often assumed to be thermal (see \autoref{subsec:collisional_excitation}), the distribution of electron velocities can be approximated by a Maxwell-Boltzmann distribution. Integrating \autoref{eq:bremsstrahlung_single} over \autoref{eq:maxwellian} gives the free-free emission produced by a thermal distribution of electrons as a function of temperature,
\begin{equation}\label{eq:bremsstrahlung}
    P_{ff}(\lambda,T_e) = \frac{32\pi e^6}{3m_ec^3}\left(\frac{2\pi}{3k_Bm_e}\right)^{1/2}\frac{Z^2}{\lambda^2T_e^{1/2}}n_en_i\exp{\left(-\frac{hc}{\lambda k_BT_e}\right)}\langle g_{ff}\rangle,
\end{equation}
where $\langle g_{ff}\rangle$ is the velocity-averaged Gaunt factor and is, in general, nontrivial to calculate \citep{rybicki_radiative_1979}. \citet{itoh_relativistic_2000} provide an analytical fitting formula for the relativistic, velocity-averaged Gaunt factor which can be used to evaluate $\langle g_{ff}\rangle$ for the conditions $10^6\le T_e\le10^{8.5}$ \si{\kelvin} and $-4\le\log{(hc/\lambda k_BT_e)}\le1$. Otherwise, the non-relativistic Gaunt factors of \citet{sutherland_accurate_1998} can be used. \autoref{fig:free-free} shows the free-free emission summed over all the ions of Fe as a function of $\lambda$ for three different temperatures: \SIlist{1;10;100}{\mega\kelvin}. Note that $P_{ff}$ decays exponentially with increasing $\lambda$ and increases with increasing $T_e$.

% spell-checker: disable %
\begin{pycode}[chapter3]
fe = fiasco.Element('Fe',[1e6,1e7,1e8]*u.K)
w = np.logspace(-1.5,3,1000)*u.angstrom
brem = fe.free_free(w)
fig = plt.figure(figsize=texfigure.figsize(pytex,scale=1,))
ax = fig.gca()
for i in range(brem.shape[0]):
    ax.plot(w, brem[i,:], 
            label=f'\SI{{{fe.temperature[i].to(u.MK).value:.0f}}}{{\mega\kelvin}}',
            color=PALETTE[i])
ax.set_yscale('log')
ax.set_ylim(1e-30,2e-26)
ax.set_xscale('log')
ax.set_xlim(w[[0,-1]].value)
ax.set_xlabel(r'$\lambda$ $[\si{\angstrom}]$')
ax.set_ylabel(r'$P_{ff}$ $[\si{\erg\cubic\cm\per\second\per\angstrom}]$')
ax.legend(frameon=False,loc=1)
tfig = ch3.save_figure('free-free', fext='.pgf')
tfig.caption = r'Free-free emission summed over all ions of Fe as a function of wavelength. The different curves correspond to \SI{1}{\mega\kelvin} (blue), \SI{10}{\mega\kelvin} (orange), and \SI{100}{\mega\kelvin} (green). The factor $n_en_i$ is not included here such that $P_{ff}$ has no density dependence.'
\end{pycode}
\py[chapter3]|tfig|
% spell-checker: enable %

\subsection{Free-bound Emission}

In addition to bremsstrahlung, \textit{free-bound} emission, wherein a free electron is captured by an ion and produces a photon, can also contribute to the continuum. This mechanism can be expressed as,
\begin{equation}\label{eq:free_bound_process}
    X_{k+1} + e(E) \to X_{k} + h\nu.
\end{equation}
Note that the photon produced has energy $h\nu=E+I$, where $I$ is the ionization energy of the bound state of the captured electron \citep{del_zanna_solar_2018}. The emissivity due to free-bound emission assuming a Maxwellian distribution of electron velocities is given by Equation 12 of \citet{young_chianti-atomic_2003}. The free-bound continuum spectra is characterized by sharp discontinuities at the ionization threshold of the ion because an electron with energy greater than the ionization threshold will not be captured by the ion. 

Two-photon decay processes in ions in the H and He isoelectronic sequences may also contribute to the total continuum emission \citep{young_chianti-atomic_2003}. However, compared to bremsstrahlung and free-found emission, the contribution of the two-photon decay process to the total continuum emission is relatively small for $T_e\SI{\le 1e4}{\kelvin}$ \citep{del_zanna_solar_2018}. The continuum emission due to free-free, free-bound, and two-photon decay processes can all be computed using data and associated software provided by CHIANTI.

\section{Temperature Sensitivity of the AIA Passbands}\label{sec:aia_response}

The primary focus of this thesis is the prediction and analysis of observations from the Atmospheric Imaging Assembly (AIA) instrument. Thus, before discussing the differential emission measure (\autoref{sec:dem}) and time lag diagnostics (\autoref{sec:timelag}), it is important to understand the temperature sensitivity of the EUV passbands of the AIA instrument. The AIA instrument onboard the SDO spacecraft is comprised of four dual-channel normal-incidence telescopes which have continuously observed the full Sun (\SI{41}{\arcminute} field of view) since 29 April 2010 at a cadence of \SIrange{10}{12}{\second} and spatial resolution of $\approx\SI{0.6}{\arcsecond}$ per pixel \citep{lemen_atmospheric_2012,boerner_initial_2012}. In addition to two ultraviolet (UV, \SIlist{1600;1700}{\angstrom}) channels and one visible (\SI{4500}{\angstrom}) channel, AIA has seven extreme ultraviolet (EUV) channels: \SIlist{94;131;171;193;211;304;335}{\angstrom}. Unlike the other EUV channels which are primarily sensitive to hot ($T\ge10^{5.8}$ \si{\kelvin}) lines in high ionization states of Fe, the \SI{304}{\angstrom} channel is dominated by the much cooler \SI{303.784}{\angstrom} He II line \citep[see Table 1 of][]{lemen_atmospheric_2012}. This line forms primarily in the chromosphere and is optically-thick, meaning its observed intensity is not well-modeled by \autoref{eq:intensity} and data in the CHIANTI database \citep{boerner_initial_2012,warren_solar_2005}. Because the primary focus of this thesis is the dynamics of hot plasma in \AR s, diagnostics will be limited to the six remaining EUV channels: \SIlist{94;131;171;193;211;335}{\angstrom}. \autoref{tab:aia_channels} summarizes the solar features, dominant Fe ions, and characteristic temperature of each of channel.

\begin{table}
    \centering
    \caption{Primary ions observed by the six AIA EUV channels of interest. Adapted from Table 1 of \citet{lemen_atmospheric_2012}.\label{tab:aia_channels}}
    \begin{tabularx}{\columnwidth}{cclC}
        \toprule
        Channel $[\si{\angstrom}]$ & Primary ion(s) & Solar feature & Characteristic temperature $[\si{\kelvin}]$ \\
        \midrule
        94 & Fe XVIII & flaring corona & $10^{6.8}$ \\
        131 & Fe VIII, XXI & TR, flaring corona & $10^{5.6},10^7$ \\
        171 & Fe IX & quiet corona, upper TR & $10^{5.8}$ \\
        193 & Fe XII, XXIV & corona, hot flare plasma & $10^{6.2},10^{7.3}$ \\
        211 & Fe XIV & active region & $10^{6.3}$ \\
        335 & Fe XVI & active region & $10^{6.4}$ \\
        \bottomrule
    \end{tabularx}
\end{table}

For any of the aforementioned EUV channels of AIA, the intensity as observed by a particular channel $c$ can be written as,
\begin{equation}\label{eq:aia_intensity}
    p_c = \int_\textup{LOS}\dd{h}n_Hn_eK_c(T_e),
\end{equation}
where $K_c$ is the temperature response function of channel $c$ and is given by,
\begin{equation}\label{eq:temperature_response}
    K_c = \int_0^\infty\dd{\lambda}G(\lambda)R_c(\lambda),
\end{equation}
where $G(\lambda)$ is the total contribution function of all ions and $R_c$ is the wavelength response of channel $c$ \citep{boerner_initial_2012}. Note that \autoref{eq:aia_intensity} gives the intensity in units of \si{\dn\per\pixel\per\second} such that $K_c$ has units \si{\dn\per\pixel\per\second\cm\tothe{5}} and $R_c$ has units \si{\cm\squared\dn\per\photon\steradian\per\pixel}. The wavelength response function incorporates all of the properties of the telescope channel, including the geometrical collecting area, the reflectance of the mirrors, the transmission efficiency of the filters, the quantum efficiency of the CCD, and the plate scale \citep[see Section 2 and Table 2 of][]{boerner_initial_2012}. An additional correction is applied to account for the degradation of the instrument over time. The wavelength response functions for the six EUV channels of interest are shown in \autoref{fig:aia-wavelength-response}.

% spell-checker: disable %
\begin{pycode}[chapter3]
aia = InstrumentSDOAIA([0,1]*u.s, None)
fig,axes = plt.subplots(2,3,sharey=True,
                        figsize=texfigure.figsize(pytex,scale=1,height_ratio=2/3))
for c,ax in zip(aia.channels, axes.flatten()):
    wave = np.linspace(c['wavelength'].value-10, c['wavelength'].value+10, 100)*u.angstrom
    r = splev(wave.value, c['wavelength_response_spline'])
    ax.plot(wave, r/r.max())
    ax.set_xlim(c['wavelength'].value-10, c['wavelength'].value+10)
    ax.text(c['wavelength'].value+4, 0.85, f'{c["wavelength"].value:.0f} '+ r'$\si{\angstrom}$',
            fontsize=plt.rcParams['legend.fontsize'])
axes[1,0].set_xlabel(r'$\lambda$ $[\si{\angstrom}]$')
axes[1,0].set_ylabel(r'$R_c/\max{R_c}$')
axes[1,0].set_ylim(0,1.05)
tfig = ch3.save_figure('aia-wavelength-response', fext='.pgf')
tfig.caption = r'AIA wavelength response functions for the six primary EUV channels. For each channel, the response function is shown at $\pm\SI{10}{\angstrom}$ of the nominal wavelength. Each response function is normalized to the maximum value of $R_c$ over this interval.'
\end{pycode}
\py[chapter3]|tfig|
% spell-checker: enable %

While $R_c$ is only a function of the properties of the instrument, the temperature sensitivity of each channel, $K_c$ (\autoref{eq:temperature_response}), has an explicit dependence on the atomic data through the total contribution function, $G(\lambda)$. $G(\lambda)$ includes components from both line and continuum emission such that \autoref{eq:temperature_response} can be expressed more practically as,
\begin{equation}\label{eq:temperature_response_discrete}
    K_c = \int_0^\infty\dd{\lambda}P_{\textup{cont.}}(\lambda)R_c(\lambda)  + \sum_{\{ji\}}\,\frac{G_{ji}}{hc/\lambda_{ji}}R_c(\lambda_{ji}),
\end{equation}
where $P_{\textup{cont.}}$ is the continuum emissivity (see \autoref{sec:continuum}) and $G_{ji}$ is the contribution function for an atomic transition $\lambda_{ji}$ (\autoref{eq:contribution_function}). The sum is taken over $\{ji\}$, the set of all known atomic transitions for the relevant wavelength range of channel $c$. Note that $hc/\lambda_{ji}$ is the energy per photon of wavelength $\lambda_{ji}$ such that $G_{ji}/(hc/\lambda_{ji})$ has units of \si{\photon\cubic\cm\per\second}. The set of relevant atomic transitions, $\{ji\}$, can be obtained from the CHIANTI database such that accurately determining the temperature sensitivity of the passbands depends critically on knowledge of the relevant atomic transitions in the bandpass of each channel.

% spell-checker: disable %
\begin{pycode}[chapter3]
fig = plt.figure(figsize=texfigure.figsize(pytex,scale=1,))
ax = fig.gca()
T = np.logspace(4,9,1000)*u.K
for i,c in enumerate(aia.channels):
    r = splev(T.value,c['temperature_response_spline'])
    ax.plot(
        T, r,
        label=f'{c["wavelength"].value:.0f} '+ r'$\si{\angstrom}$',
        color=PALETTE[i])
ax.set_xscale('log')
ax.set_yscale('log')
ax.set_xlim(1e5,1e8)
ax.set_ylim(1e-28,2e-24)
ax.set_xlabel(r'$T$ $[\si{\kelvin}]$')
ax.set_ylabel(r'$K_c$ $[\si{\dn\per\pixel\per\second\cm\tothe{5}}]$')
ax.legend(loc=1,frameon=False,ncol=2)
tfig = ch3.save_figure('aia-temperature-response', fext='.pgf')
tfig.caption = r'Temperature response functions for the six EUV channels of AIA listed in \autoref{tab:aia_channels} as computed by \autoref{eq:temperature_response_discrete}. Together, these six channels provide observational coverage over the temperature range $3\times10^5\lesssim T\lesssim2\times10^7$ \si{\kelvin}.'
\end{pycode}
\py[chapter3]|tfig|
% spell-checker: enable %

The temperature response functions for the EUV channels of AIA are shown in \autoref{fig:aia-temperature-response}. Combined, the six response functions provide temperature coverage over the range $3\times10^5\lesssim T\lesssim2\times10^7$ \si{\kelvin}. Note that several of the response functions, particularly the \SI{94}{\angstrom} and \SI{131}{\angstrom} channels, are double-peaked in temperature due to the finite width of the wavelength responses. This degeneracy makes associating observations with specific temperatures very difficult. For example, intensity in the \SI{131}{\angstrom} channel may correspond to very hot, \SI{> e7}{\kelvin} plasma or much cooler, \SI{< e6}{\kelvin} plasma.

\section{The Differential Emission Measure Distribution}\label{sec:dem}

% spell-checker: disable %
\begin{pycode}[chapter3dem]
name = 'chapter3'
ch3_dem = texfigure.Manager(
    pytex,
    os.path.join('.', name),
    number=3,
    **{k: os.path.join('.', name, v) for k,v in manager_opts.items()}
)
import hissw
from run_ebtel import run_ebtel
from dem import calculate_em, fit_slope, plot_hist
\end{pycode}
% spell-checker: enable %

\autoref{eq:intensity}, the intensity for a transition $\lambda_{ji}$, can be alternatively expressed as,
\begin{equation}\label{eq:intensity_dem}
    I(\lambda_{ji}) = \int\dd{T_e}\dem G_{ji},
\end{equation}
where $G_{ji}$, the so-called contribution function\footnote{Several variants of $G_{ji}$ exist in the literature such as dropping the factor of $1/4\pi$, including the $n_H/n_e\approx0.83$ term, or dropping the abundance factor $\textup{Ab}(X)$. See section 3.1 of \citet{del_zanna_solar_2018}.}, is given by,
\begin{equation}\label{eq:contribution_function}
    G_{ji} = \frac{1}{4\pi}\frac{hc}{\lambda_{ji}}\textup{Ab}(X)\frac{f_{X,k}N_{X,k,j}A_{ji}}{n_e}.
\end{equation}
$\dem$ is the \textit{differential emission measure distribution} and is defined as,
\begin{equation}\label{eq:dem}
    \dem = n_en_H\left(\frac{dT_e}{dh}\right)^{-1},
\end{equation}
where $\frac{dT_e}{dh}$ is the temperature gradient along the LOS. Qualitatively, the $\dem$ measures the amount of material along the LOS between temperatures $T_e$ and $dT_e$ \citep{withbroe_thermal_1978}. Additionally, for a particular temperature interval $\Delta$, the \textit{emission measure distribution} at a particular temperature $T_{e,j}$ is defined as,
\begin{equation}\label{eq:em}
    \emd[T_{e,j}] = \int_{T_{e,j} - \Delta/2}^{T_{e,j} + \Delta/2}\dd{T_e}\dem,
\end{equation}
\citep{del_zanna_solar_2018}. For an unresolved point source (e.g. a star other than the Sun), the integration in \autoref{eq:intensity} is often taken over a volume $V$ such that $\dem$ and $\emd$ are defined in terms of $dV$ rather than $dh$. Because of this ambiguity, \autoref{eq:dem} and \autoref{eq:em} are sometimes referred to as the column differential emission measure and column emission measure, respectively.

Because the $\dem$ (and $\emd$) describes the distribution of emitting material in temperature space, it is a useful diagnostic for understanding the underlying thermodynamics of the coronal plasma. This is best illustrated by example. The left panel of \autoref{fig:em-slope} shows the electron temperature, $T_e$, for a \SI{40}{\mega\m} loop heated by a single nanoflare at $t=\SI{0}{\second}$ (blue) and 10 nanoflares every \SI{300}{\second} (orange) as simulated by the two-fluid EBTEL model (\autoref{sec:ebtel}). In both cases, the duration of each nanoflare is \SI{200}{\second} and the total energy deposited in the loop is \SI{10}{\erg\per\cubic\cm}. In the single-nanoflare case, the loop is rapidly heated to \SI{> 10}{\mega\kelvin} and then allowed to cool, uninterrupted, by thermal conduction and radiation down to its equilibrium temperature, \SI{< 1}{\mega\kelvin}. In contrast, the loop heated by 10 nanoflares is never allowed to cool below \SI{3}{\mega\kelvin} because the time between consecutive nanoflares is longer than the fundamental cooling time of the loop. Additionally, the loop is only heated to a maximum temperature of just over \SI{4}{\mega\kelvin} following each event because the energy of each nanoflare is $1/10$ of the single nanoflare energy. Thus, the loop heated by a single nanoflare samples a wide range of temperatures while the loop heated by 10 nanoflares is kept within a very narrow range in $T_e$, centered near \SI{4}{\mega\kelvin}.

The right panel of \autoref{fig:em-slope} shows the corresponding emission measure distributions, $\emd$, for the two loops. Note that compared to $\emd$ for the loop heated by 10 nanoflares, which is narrowly peaked in th range \SIrange{3}{4}{\mega\kelvin}, the $\emd$ for the loop heated by a single nanoflare is very broad because the loop samples a much larger temperature range. Thus, for infrequently (relative to the fundamental loop cooling time) heated loops, a broad $\emd$ is expected while frequently heated loops lead to a more narrow, isothermal $\emd$.

% spell-checker: disable %
\begin{pycode}[chapter3dem]
fig,ax = plt.subplots(
    1,2,figsize=texfigure.figsize(pytex,scale=1, height_ratio=0.5))
# Configure EBTEL simulation
Heq = 2e-3
duration = 200.
config = {
    'total_time': 5e3,
    'tau': 1,
    'tau_max': 10,
    'loop_length': 4e9,
    'saturation_limit': 1,
    'force_single_fluid': False,
    'use_c1_loss_correction': True,
    'use_c1_grav_correction': True,
    'use_flux_limiting': True,
    'use_power_law_radiative_losses': True,
    'calculate_dem': False,
    'save_terms': False,
    'use_adaptive_solver': True,
    'adaptive_solver_error': 1e-10,
    'adaptive_solver_safety': 0.5,
    'c1_cond0': 2.0,
    'c1_rad0': 0.6,
    'helium_to_hydrogen_ratio': 0.075,
    'surface_gravity': 1.0,
    'heating': OrderedDict({
        'partition': 1.,
        'background': 3.5e-5,}),
}

# Single nanoflare
N = 1
tn = (config['total_time'] - N*duration)/N
events = []
for i in range(N):
    events.append({'event':{
        'rise_start': i*(tn + duration),
        'rise_end': i*(tn + duration) + duration/2,
        'decay_start': i*(tn + duration) + duration/2,
        'decay_end': i*(tn + duration) + duration,
        'magnitude': 2*config['total_time']*Heq/N/duration,}})

config['heating']['events'] = events
res = run_ebtel(config, EBTEL_DIR)
ax[0].plot(res['time'], res['electron_temperature'].to(u.MK), color=PALETTE[0])
bins_T, H = calculate_em(
    res['time'], res['electron_temperature'], res['density'], config['loop_length'])
plot_hist(ax[1], H.value, bins_T.value, color=PALETTE[0])
a, b, T_fit = fit_slope(bins_T, H, T0=1.25e6*u.K, T1=4e6*u.K)
ax[1].plot(T_fit, b*T_fit**a, color=PALETTE[0], ls='--', label=f'$a={a:.1f}$')
# Repeated nanoflares
N = 10
tn = (config['total_time'] - N*duration)/N
events = []
for i in range(N):
    events.append({'event':{
        'rise_start': i*(tn + duration),
        'rise_end': i*(tn + duration) + duration/2,
        'decay_start': i*(tn + duration) + duration/2,
        'decay_end': i*(tn + duration) + duration,
        'magnitude': 2*config['total_time']*Heq/N/duration,}})

config['heating']['events'] = events
res = run_ebtel(config, EBTEL_DIR)
ax[0].plot(res['time'], res['electron_temperature'].to(u.MK), color=PALETTE[1])
bins_T, H = calculate_em(
    res['time'], res['electron_temperature'], res['density'], config['loop_length'])
plot_hist(ax[1], H.value, bins_T.value, color=PALETTE[1])
a, b, T_fit = fit_slope(bins_T, H, T0=1.25e6*u.K, T1=4e6*u.K)
ax[1].plot(T_fit, b*T_fit**a, color=PALETTE[1], ls='--', label=f'$a={a:.1f}$')
# Labels and limits
ax[0].set_xlim(0,5e3)
ax[0].set_ylim(0.3, 11.5)
ax[0].set_xlabel(r'$t$ $[\si{\second}]$')
ax[0].set_ylabel(r'$T_e$ $[\si{\mega\kelvin}]$')
ax[1].set_xscale('log')
ax[1].set_yscale('log')
ax[1].set_ylim(5e25, 2e28)
ax[1].set_xlim(3e5, 2e7)
ax[1].set_xlabel(r'$T_e$ $[\si{\kelvin}]$')
ax[1].set_ylabel(r'EM $[\si{\cm\tothe{-5}}]$')
ax[1].legend(loc=2, frameon=False)
plt.subplots_adjust(wspace=0.35)
# Save
tfig = ch3_dem.save_figure('em-slope', fext='.pgf')
tfig.caption = r'\textit{Left panel:} Two-fluid EBTEL simulations of the electron temperature, $T_e$, for a loop heated by a single nanoflare at $t=\SI{0}{\second}$ (blue) and loop heated by 10 nanoflares every \SI{300}{\second} (orange) for a total simulation time of \SI{5e3}{\second}. In both cases, the loop length is \SI{40}{\mega\m}, the duration of each nanoflare is \SI{200}{\second}, and the total energy deposited in the electrons in the loop is \SI{10}{\erg\per\cubic\cm}. \textit{Right panel:} $\emd$ for the two loops shown in the left panel. The dashed lines denote the power-law fit, $T_e^a$, to each distribution over the interval $\SI{1.25}{\mega\kelvin}\le T_e\le\SI{4}{\mega\kelvin}$ and the emission measure slopes are shown in the legend. $\emd$ is approximated by binning $T_{e,i}$, weighted by $n_i^2L$, at each timestep $t_i$ in temperature bins between \SIrange{e4}{e8.5}{\kelvin} with width 0.05 in $\log{T_e}$ and then time-averaging over the whole simulation. Note that both distributions peak at approximately the same temperature.'
\end{pycode}
\py[chapter3dem]|tfig|
% spell-checker: enable %

\subsection{The Emission Measure Slope}\label{sec:em_slope}

\citet{jordan_structure_1975,jordan_structure_1976} found that for $T_e\approx$\SIrange{e5}{e6}{\kelvin}, the observed emission measure distribution could be described by,
\begin{equation}\label{eq:em_slope}
    \emd \propto T_e^a,
\end{equation}
where $a$ is some constant. This implies $\log{\emd}$ is linear in $\log{T_e}$ over the given temperature range with slope $a$ such that $a$ is often referred to as the \textit{emission measure slope}. \citet{athay_theoretical_1966,jordan_energy_1980} found that a slope of $a=3/2$ was most consistent with observations of the quiet Sun. \citet{cargill_implications_1994} simulated loops heated by nanoflares using a loop cooling model and found that this scaling also held between \SI{e6}{\kelvin} and \SI{e6.4}{\kelvin} with $a\approx4$. Generally speaking, \autoref{eq:em_slope} is applicable for temperatures ``coolward'' of $T_{peak}=\argmax_{T_e}\emd$ down to $\approx\SI{e5}{\kelvin}$.

Using simple scaling laws, one can derive an expression for $a$ for a loop undergoing radiative cooling. The emission measure distribution can be approximated as $\emd\sim n^2\tau_\textup{rad}$ \citep{cargill_implications_1994}, where $\tau_\textup{rad}\sim T_e^{1-\alpha}n^{-1}$ is the radiative cooling time and $\alpha$ controls the temperature-dependence of the radiative loss function (see \autoref{ch:loops}). Assuming $T_e\propto n^{l}$ gives,
\begin{equation}
    \emd \propto T_e^{1/l + 1 - \alpha}.
\end{equation}
While the empirical result $l\approx2$ of \citet{serio_dynamics_1991,jakimiec_dynamics_1992} is often used, \citet{bradshaw_cooling_2010} showed that accounting for the enthalpy flux during radiative cooling leads to $l\sim1$ for long loops and $l\sim2$ for short loops. Using $\alpha=-1/2$ \citep[from the scaling laws of][see \autoref{sec:scaling_laws}]{rosner_dynamics_1978}, one finds $a=5/2$ for long loops and $a=2$ for short loops. The dashed lines in the right panel of \autoref{fig:em-slope} show \autoref{eq:em_slope} fit over the temperature range $\SI{1.25}{\mega\kelvin}\le T_e\le\SI{4}{\mega\kelvin}$ and the values of the emission measure slope are shown in the legend. Note that $a=2$ for the single nanoflare case as expected from the analytical scaling while $a=12.4$ for the repeating nanoflare case because the loop is never allowed to undergo full radiative cooling before being reheated. 

Many workers have used the emission measure slope to infer properties of the underlying heating from the observed $\emd$. These studies are summarized in \autoref{tab:em_slope}. Notably, \citet{warren_systematic_2012} computed $\emd$ distributions for the cores of 15 different \AR s using spectroscopic observations of 22 different lines from \textit{Hinode}/EIS and narrowband intensities from the \SI{94}{\angstrom} channel of AIA. \citeauthor{warren_systematic_2012} found $2\le a\le 4.8$ and that shallower slopes coincided with lower magnetic flux. Additionally, \citet{cargill_active_2014} systematically addressed the relationship between the frequency of energy deposition and the emission measure slope by computing $a$ for a range of waiting times between consecutive nanoflare events using the EBTEL loop model. \citeauthor{cargill_active_2014} found that $a$ became large ($\gtrsim8$) for very short waiting times while $a\approx2$ for waiting times greater than the characteristic loop cooling time, and that a scaling between the nanoflare energy and the waiting time was needed to reproduce the range of observed slopes. While the emission measure slope is a relatively simple and easily interpretable diagnostic, one should exercise caution when comparing slopes derived from observations to those from models as uncertainties in the atomic data can lead to confidence intervals of at least \numrange{1}{1.3} on $a$ \citep{guennou_can_2013}.

\begin{table}[!h]
\begin{threeparttable}
    \centering
    \caption{Summary of observational and modeling studies that have used the emission measure slope, $a$, as a diagnostic for the underlying energy deposition. The approximate range of observed slopes is $2\lesssim a\lesssim5$. Adapted from Table 3 of \citet{bradshaw_diagnosing_2012}.\label{tab:em_slope}}
    \renewcommand{\arraystretch}{1.2}
    \begin{tabularx}{\columnwidth}{lcCC}
        \toprule
        Reference & Type & Slope ($a$) & Temperature range $[\si{\kelvin}]$ \\
        \midrule
        \citet{warren_constraints_2011} & observation & 3.26 & \numrange[range-phrase = --]{e6}{e6.6} \\
        & model & 2.17 &  \\
        \citet{winebarger_using_2011} & observation & 3.2 & \numrange[range-phrase = --]{e6}{e6.5} \\
        \citet{tripathi_emission_2011} & observation & \numrange[range-phrase = --]{2.08}{2.47} & \numrange[range-phrase = --]{e5.5}{e6.55} \\
        & & \numrange[range-phrase = --]{2.05}{2.7}\tnote{a} & \\
        \citet{mulu-moore_can_2011} & model\tnote{b} & \numrange[range-phrase = --]{1.6}{2} & $10^6$--$T_{peak}$\tnote{c} \\
        & & \numrange[range-phrase = --]{2}{2.3} & \\
        \citet{warren_systematic_2012} & observation & \numrange[range-phrase = --]{1.7}{4.5} & \numrange[range-phrase = --]{e6}{e6.6} \\
        \citet{schmelz_cold_2012} & observation & \numrange[range-phrase = --]{1.91}{5.17} & $10^6$--$T_{peak}$\tnote{d} \\
        \citet{bradshaw_diagnosing_2012} & model & \numrange[range-phrase = --]{0.81}{2.56} & $10^6$--$T_{peak}$\tnote{e} \\
        \citet{reep_diagnosing_2013} & model & \numrange[range-phrase = --]{0.88}{4.56} & $10^6$--$T_{peak}$\tnote{f} \\
        \citet{cargill_active_2014} & model & \numrange[range-phrase = --]{2}{8} & $T_0$--$10^{6.6}$\tnote{g} \\
        \citet{del_zanna_evolution_2015} & observation\tnote{h} & $4.4\pm0.4$ & \numrange[range-phrase = --]{e6}{3e6} \\
        & & $4.6\pm0.4$ & \\
        \bottomrule
    \end{tabularx}
    \begin{tablenotes}
        \footnotesize
        \item[a] $\dem$ computed from background-subtracted observations.
        \item[b] Intensities were modeled using photospheric (first row) and coronal (second row) abundances.
        \item[c] $T_{peak}$ varied from \SIrange{e6.6}{e6.8}{\kelvin}.
        \item[d] $T_{peak}$ varied from \SIrange{e6.3}{e6.8}{\kelvin}.
        \item[e] $T_{peak}$ varied from \SIrange{e5.85}{e7.35}{\kelvin}.
        \item[f] $T_{peak}$ varied from \SIrange{e6.35}{e6.65}{\kelvin}.
        \item[g] $a$ is computed for 12 different values of $T_0$ between \numrange[range-phrase = and]{e6}{e6.25} and averaged.
        \item[h] The slope was computed in every pixel of \AR{} NOAA 11193 once when it first appeared (first row) and then again after one rotation (second row). 
    \end{tablenotes}
\end{threeparttable}
\end{table}

The $\emd$ distribution ``hotward'' of $T_{peak}$ provides a potential diagnostic for the nanoflare model \citep{cargill_implications_1994,cargill_nanoflare_2004} and a similar scaling to \autoref{eq:em_slope}, $\emd\propto T_e^{-b}$, has been claimed for $T_e>T_{peak}$ \citep[e.g.][]{warren_systematic_2012}. However, the amount of emission in this temperature range is not well constrained by current observations \citep{winebarger_defining_2012}. Furthermore, $b$ is very sensitive to the temperature range over which it is calculated such that it is not a robust diagnostic for characterizing the amount of very hot plasma \citep{barnes_inference_2016-1}. I will defer a detailed discussion of the ``hot'' part of the emission measure distribution to \autoref{ch:inferring_hot_plasma}.

\subsection{Determining the DEM from Observations}

Though a useful thermal diagnostic, deriving the differential emission measure distribution from observed spectral line or narrowband intensities is very difficult and can yield ambiguous results. \autoref{eq:intensity_dem} is an inhomogeneous Fredholm equation of the first kind,
\begin{equation}\label{eq:fredholm}
    g(t) = \int_a^b\dd{s}K(t,s)f(s),
\end{equation}
where $g(t)$ is some measured value, $K(t,s)$ is the kernel, and $f(s)$ is the unknown function to be determined \citep{press_numerical_1992}. Comparing with \autoref{eq:intensity_dem}, $s$ and $t$ correspond to the temperature and wavelength or channel, respectively. In principle, solving \autoref{eq:fredholm} to find $f$ requires inverting the kernel matrix provided $K$ is invertible. However, this is complicated by the fact that information about $f$ (or in the case of \autoref{eq:intensity_dem}, $\dem$) is lost when $f$ is ``smoothed'' by the kernel $K$ such that solutions to $f$ derived by inverting $K$ will be extremely sensitive to uncertainties in the input $g$ \citep{press_numerical_1992}. In applied mathematics, there exists a great deal of literature on solution methods for these so-called \textit{inverse problems}, of which \autoref{eq:intensity_dem} is a prime example.

In solar astrophysics specifically, a great amount of effort has been focused on methods for solving \autoref{eq:intensity_dem} for the $\dem$ ($f$) given a set of observed EUV and/or X-ray intensities ($g$) using contribution functions or detector responses ($K$) derived from atomic data. These methods include, but are not limited to, Gaussian, multi-Gaussian, and spline fits via $\chi^2$-minimization \citep[e.g.][]{guennou_accuracy_2012,warren_observations_2013,ryan_compatibility_2014,caspi_constraining_2014}, Markov Chain Monte Carlo \citep[MCMC,][]{kashyap_markov-chain_1998}, regularized inversion \citep{hannah_differential_2012,plowman_fast_2013}, sparse basis pursuit \citep{cheung_thermal_2015}, and sparse Bayesian inference \citep{warren_sparse_2017}. Many of these methods have been designed to work well with specific types of observations and, due to the lack of a unique solution to \autoref{eq:intensity_dem}, tend to all give somewhat different $\dem$ for the same set of inputs $I$ \citep[see comparison of 15 different methods by][]{aschwanden_benchmark_2015}.

In addition to the mathematical difficulties, uncertainties in the atomic data (e.g. due to line blends, line misidentification) make interpreting the $\dem$ even more ambiguous though there has been recent progress in understanding how these uncertainties propagate through to plasma diagnostics \citep[e.g.][]{guennou_can_2013,yu_incorporating_2018,del_zanna_uncertainties_2019}. Additionally, ions in the Li and Na isoelectronic sequences are known to exhibit ``anomalous behavior'' such that when lines of these ions are used to constrain the $\dem$, the intensities are consistently overestimated or underestimated by up to a factor of 5 \citep{burton_w._m._discussion_1971,dupree_analysis_1972,del_zanna_spectroscopic_2002}. It has been suggested that this behavior may be due to departures from ionization equilibrium or non-Maxwellian electron distributions \citep{del_zanna_solar_2018}. Thus, when selecting observed lines to constrain the $\dem$, care should be taken to avoid these anomalous ions as well as those ions whose contribution functions have a strong density dependence \citep[e.g. Fe IX,][]{del_zanna_solar_2018}. It should be noted that several authors \citep{craig_fundamental_1976,judge_failure_1995,judge_fundamental_1997,judge_coronal_2010} claim the uncertainties in the atomic data and observations combined with the ill-posed nature of \autoref{eq:intensity_dem} prevent a determination of a temperature distribution to within a degree of certainty that is of physical interest.

\subsubsection{Emission Measure Loci}

Though inverting \autoref{eq:intensity_dem} to find a full solution for the $\dem$ is difficult, an upper bound on $\emd$ can be estimated using the emission measure loci method first developed by \citet{veck_solar_1981}. For an isothermal plasma at temperature $T_c$, \autoref{eq:intensity_dem} becomes,
\begin{equation}
    I(\lambda_{ji}) = G_{ji}(T_c)\int_{LOS}\dd{h}n_e^2 = G_{ji}(T_c)\textup{EM}.
\end{equation}
Given this approximation, one can define a function called the emission measure loci,
\begin{equation}\label{eq:em_loci}
    \emd_{\textup{loci}} = \frac{I(\lambda_{ji})}{G_{ji}(T_e)}.
\end{equation}
$\emd_{\textup{loci}}$ depends strongly on $T_e$ because of the temperature-dependence of the contribution function, $G_{ji}(T_e)$ (see \autoref{eq:contribution_function}). Note that $\emd[T_c]_{\textup{loci}}=\textup{EM}$. By plotting \autoref{eq:em_loci} as a function of $T_e$ for multiple lines and finding where all of the emission measure loci curves intersect, one can find $T_c$, the temperature of the isothermal plasma and $\textup{EM}$. This is because for $T=T_c$, the right-hand side of \autoref{eq:em_loci} should be equal for an isothermal plasma with uniform density. For a multithermal plasma, \autoref{eq:em_loci} still provides an estimate for the upper bound on $\emd$, but the $\emd_{\textup{loci}}$ will not all intersect at a single temperature \citep{phillips_ultraviolet_2008}.

\subsubsection{Regularized Inversion}\label{subsubsec:dem_regularized_inversion}

Despite the aforementioned difficulties, the differential emission measure distribution is a still a practical diagnostic for understanding how energy is deposited in the coronal plasma. The method developed by \citet{hannah_differential_2012} allows for efficient determination of the $\dem$ from AIA observations (or observed spectral line intensities) by solving \autoref{eq:intensity_dem} through \textit{regularized inversion} wherein a smoothing parameter is introduced to guarantee a well-behaved and unique solution to an otherwise ill-posed problem. In the work presented in this thesis, I will compute the differential emission measure distribution from both predicted and observed AIA intensities using the method of \citet{hannah_differential_2012}. I will now describe this approach for computing the $\dem$ and provide an example.

Using the notation of \citet{hannah_differential_2012}, \autoref{eq:intensity_dem} can be written in matrix form as,
\begin{equation}\label{eq:intensity_dem_matrix}
    \mathbf{g} = \mathbf{K} \bm{\xi} + \delta\mathbf{g},
\end{equation}
where $\mathbf{g}$ is a vector of intensities for each EUV channel, $\delta\mathbf{g}$ are the associated uncertainties, $\mathbf{K}$ is the matrix 
of temperature response functions\footnote{Note that the contribution function has been replaced by the temperature response function in the kernel as $\mathbf{g}$ denotes a narrowband intensity rather than spectral line emission. The temperature response function is effectively the contribution function for the detector channel.} (see \autoref{sec:aia_response}) for each channel as a function of $T_e$, and $\bm{\xi}$ is the differential emission measure as a function of $T_e$. Recovering $\bm{\xi}$ by simply inverting $\mathbf{K}$ is not possible due to noise amplification from the uncertainties in the data $\delta\mathbf{g}$. Additionally, this system may be under-determined if the number of temperature bins is greater than the number of detector channels. One common method for dealing with these difficulties is to add linear constraints to $\bm{\xi}$ in the form of zeroth-order regularization \citep[e.g.][]{tikhonov_solution_1963}. The regularized least squares problem for \autoref{eq:intensity_dem_matrix} then becomes,
\begin{equation}\label{eq:regularized_least_squares}
    \textup{minimize} \quad \norm{\tilde{\mathbf{K}}\bm{\xi} - \tilde{\mathbf{g}}} + \lambda\norm{\mathbf{L}(\bm{\xi} - \bm{\xi}_0)},
\end{equation}
where $\tilde{\mathbf{K}}=(\delta\mathbf{g})^{-1}\mathbf{K}$, $\tilde{g}=(\delta\mathbf{g})^{-1}\mathbf{g}$, $\norm{\mathbf{x}}=\sum_ix_i^2$ is the $\ell^2$ norm, $\mathbf{L}$ is the constraint matrix, $\lambda$ is the regularization parameter, and $\bm{\xi}_0$ is the initial guess of the solution. Solutions to \autoref{eq:regularized_least_squares} are unique and well-behaved and can be expressed in terms of $\lambda$. The exact value of $\lambda$ depends on the desired $\chi^2$ of the solution and may need to be ``tweaked'' depending on the input data. An additional constraint may be added to ensure the positivity of $\bm{\xi}$.

% spell-checker: disable %
\begin{pycode}[chapter3dem]
# Create DEM
T0 = 10**(6.5)*u.K
sigma = 0.15
dem_total = 3.76e22*u.cm**(-5)*u.K**(-1)
T = 10**np.arange(5.5,7.5,0.01)*u.K
dem_true = dem_total/(np.sqrt(2*np.pi)*sigma)*np.exp(-((np.log10(T.value) - np.log10(T0.value))**2)/(2*sigma**2))
# Calculate intensities
aia = InstrumentSDOAIA([0,1]*u.s,None)
I_true = [(dem_true.value
           *splev(T.value, c['temperature_response_spline'])
           *np.gradient(T).value).sum()
          for c in aia.channels]
# Run regularized inversion code
ssw = hissw.ScriptMaker(
    ssw_packages=['sdo/aia'],
    ssw_paths=['aia','xrt'],
    extra_paths=[DEMREG_DIR]
)
script = """
data = {{data}}
; Calculate errors
nchannels = n_elements(data)
data_errors=fltarr(nchannels)
exptimes = [2.901051,2.901311,2.000197,1.999629,2.901278,2.900791]
channels = [94,131,171,193,211,335]
common aia_bp_error_common,common_errtable
common_errtable=aia_bp_read_error_table('{{ aia_error_table_filename }}')
for i=0,nchannels-1 do begin
    data_errors[i]=aia_bp_estimate_error(data[i]*exptimes[i],channels[i],n_sample=1)
    data_errors[i]=data_errors[i]/exptimes[i]
endfor
; Get temperature bins
response_logt = {{log_temperature}}
temperature = {{temperature_bin_edges}}
; Calculate response functions
response_matrix = {{ response_matrix }}
; DEM Calculation
dn2dem_pos_nb,data,data_errors,response_matrix,response_logt,temperature,dem,dem_errors,logt_errors,chi_squared,dn_regularized
"""
T_bins = 10**(np.arange(5.5,7.5,0.1))*u.K
T_bins_centers = (T_bins[1:] + T_bins[:-1])/2.
K_matrix = np.stack([
    splev(T_bins_centers.value, c['temperature_response_spline']) for c in aia.channels
])
input_args =  {
    'log_temperature': np.log10(T_bins_centers.value).tolist(),
    'temperature_bin_edges': T_bins.value.tolist(),
    'aia_error_table_filename': os.path.join(ssw.ssw_home, 'sdo/aia/response/aia_V2_error_table.txt'),
    'data': I_true,
    'response_matrix': K_matrix.tolist(),
}
save_vars = ['dem', 'dem_errors', 'logt_errors', 'data_errors']
reg = ssw.run(script, args=input_args, save_vars=save_vars,verbose=False)
# Calculate DEM intensities
I_dem = [(splev(T_bins_centers.value, c['temperature_response_spline'])
          *reg['dem']
          *np.diff(T_bins.value)).sum() for c in aia.channels]
# Plot
fig = plt.figure(figsize=texfigure.figsize(pytex,scale=1,height_ratio=0.8))
## true DEM
ax = plt.subplot2grid((4,1),(0,0),rowspan=3)
ax.plot(T, dem_true, color='k', alpha=0.5, ls='--',lw=2)
## EM loci
for i,c in enumerate(aia.channels):
    K = splev(T.value,c['temperature_response_spline'])
    ax.plot(T, I_true[i]/K/np.gradient(T)/4/np.pi, 
            label=f"{c['name']} " + r'\si{\angstrom}', color=PALETTE[i])
## regularized DEM
ax.errorbar(
    T_bins_centers.value, reg['dem'], 
    yerr=reg['dem_errors'],
    xerr=(T_bins_centers.value*(10**(reg['logt_errors']/2) - 1),
          T_bins_centers.value*(1 - 10**(-reg['logt_errors']/2))),
    color='k',marker='',ls='')
## labels and limits
ax.set_xscale('log')
ax.set_yscale('log')
ax.set_ylim(1e19,5e24)
ax.set_xlim(5e5,2e7)
ax.set_xlabel(r'$T_e$ $[\si{\kelvin}]$')
ax.set_ylabel(r'DEM $[\si{\cm\tothe{-5}\kelvin\tothe{-1}}]$')
ax.legend(frameon=False,ncol=2,loc='lower center')
## residuals
ax = plt.subplot2grid((4,1),(3,0),rowspan=1)
for i,c in enumerate(aia.channels):
    ax.plot(c['wavelength'].value, (I_true[i] - I_dem[i])/reg['data_errors'][i],
            marker='o', ls='', color=PALETTE[i])
## labels and limits
ax.set_ylim(-10,10)
ax.axhline(y=0, ls=':', color='k',lw=1)
ax.set_xlabel(r'Channel $[\si{\angstrom}]$')
ax.set_ylabel('Residuals')
plt.subplots_adjust(hspace=0.85)
# Save
tfig = ch3_dem.save_figure('dem-regularized', fext='.pgf')
tfig.caption = r'An example of the regularized inversion method of \citet{hannah_differential_2012} for a simple model $\dem$ and simulated AIA observations. The dashed, gray line is the true $\dem$, a single Gaussian pulse centered at \SI{e6.5}{\kelvin}, and the black errorbars in $T_e$ and $\dem$ denote the regularized solution. The true $\dem$ has a total emission measure of \SI{3.76e22}{\cm\tothe{-5}} and spread of $\sigma=0.15$ in $\log{T_e}$. The colored curves as given in the legend are the emission measure loci curves for each AIA EUV channel. The lower panel shows the residuals between the true and recovered intensities for each channel. Adapted from Figure 3 of \citet{hannah_differential_2012}.'
\end{pycode}
\py[chapter3dem]|tfig|
% spell-checker: enable %

\autoref{fig:dem-regularized} shows a model $\dem$ consisting of a single Gaussian pulse centered on \SI{e6.5}{\kelvin} (dashed gray) and the  regularized solution (black errorbars) as computed from the model AIA intensities using the method of \citet{hannah_differential_2012}\footnote{An  IDL implementation of the regularized inversion method of \citet{hannah_differential_2012} has kindly been made publicly available by the authors: \href{https://github.com/ianan/demreg}{github.com/ianan/demreg}}. The emission measure loci curves for each AIA channel are denoted by the colored curves. The lower panel shows the residuals between the true and recovered intensities for each channel. The uncertainties on the model intensities, $\delta\mathbf{g}$, are estimated using the error tables provided by the AIA instrument team in SSW \citep{freeland_data_1998}. Note that the regularized solution and the true $\dem$ deviate most for \SI{> e7}{\kelvin} and \SI{<7e5}{\kelvin}, where the AIA passbands are least sensitive (see \autoref{fig:aia-temperature-response}). The vertical errorbars represent the propagation of the uncertainties in the intensity through the regularized kernel matrix and are computed by taking the standard deviation of multiple Monte Carlo samples of the regularized solution from $\mathbf{g}$ in the range of $\delta\mathbf{g}$. The horizontal errors represent the best possible temperature resolution, or the temperature bias, of the method and will be larger the greater the degree of regularization. The temperature bias will also be worse if the uncertainties of the observed intensities are large.

\section{Time-lag Analysis}\label{sec:timelag}

% spell-checker: disable %
\begin{pycode}[chapter3timelag]
name = 'chapter3'
ch3_timelag = texfigure.Manager(
    pytex,
    os.path.join('.', name),
    number=3,
    **{k: os.path.join('.', name, v) for k,v in manager_opts.items()}
)
from synthesizAR.analysis import cross_correlation
from run_ebtel import run_ebtel
\end{pycode}
% spell-checker: enable %

Besides the emission measure distribution, additional diagnostics are needed in order to place tighter constraints on coronal heating properties. The time-lag method of \citet{viall_evidence_2012} is a powerful and efficient tool for understanding large-scale cooling patterns across an entire \AR{}. In the following sections, I will outline the time-lag analysis technique and explain why it is useful in discerning the underlying thermodynamic behavior of the plasma from the observed narrowband intensity. 

\subsection{Cross-correlation}\label{subsec:cross_correlation}

Broadly speaking, the time-lag method involves computing the \textit{cross-correlation} between pairs of AIA light curves. The cross-correlation measures the similarity between two signals as a function of the offset between them. Given two signals $f_A$ and $f_B$ that are a function of time $t$, the cross-correlation, $\crosscor$, can be written as,
\begin{equation}
    \crosscor = (f_A\star f_B)(\tau) = \int_{-\infty}^\infty\dd{t}f_A^{\ast}(t)f_B(t + \tau),
\end{equation}
where $\tau$ is the temporal offset between $f_A$ and $f_B$ and $f_A^{\ast}$ denotes the complex conjugate of $f_A$. Alternatively, $\crosscor$ can be expressed in terms of the convolution,
\begin{equation}\label{eq:cross_correlation_convolution}
    \crosscor = f_A(-t) \ast f_B(t),
\end{equation}
where $\ast$ denotes the convolution operator. Taking the Fourier transform of both sides of \autoref{eq:cross_correlation_convolution} and using the convolution theorem \citep[section 20.4]{arfken_mathematical_2013} gives,
\begin{align}\label{eq:cross_correlation_final}
    \fourier{\crosscor} &= \fourier{f_A(-t) \ast f_B(t)}, \nonumber \\
    \fourier{\crosscor} &= \fourier{f_A(-t)}\fourier{f_B(t)}, \nonumber \\
    \crosscor &= \inversefourier{\fourier{f_A(-t)}\fourier{f_B(t)}},
\end{align}
where $\mathscr{F}$ denotes the Fourier transform. \autoref{eq:cross_correlation_final} defines the cross-correlation $\crosscor$ between two signals $f_A$ and $f_B$ as a function of temporal offset $\tau$ in terms of the Fourier transform. Defining the cross-correlation in this manner allows one to efficiently compute $\crosscor$ using highly-optimized fast Fourier transform (FFT) algorithms \citep[e.g. the widely-used FFT algorithm developed by][]{cooley_algorithm_1965}. Note that $f_A$ and $f_B$ are standardized such that both signals have zero mean and unit standard deviation. Additionally, $\crosscor$ is scaled by the length of the signal such that the cross-correlation is always between $-1$ (perfectly anti-correlated) and $+1$ (perfectly correlated).

% spell-checker: disable %
\begin{pycode}[chapter3timelag]
# Compute cross-correlation between Gaussian pulses
def gaussian_pulse(x,x0,sigma):
    return np.exp(-(x - x0)**2/(2*sigma**2))
t = np.linspace(-100,100,100000)*u.s
fA = gaussian_pulse(t,0.25*u.s,0.05*u.s)
fB = gaussian_pulse(t,0.75*u.s,0.05*u.s)
lag,cc = cross_correlation(t,fA,fB)
# Plot
fig,axes = plt.subplots(1,2,figsize=texfigure.figsize(pytex,scale=1,height_ratio=0.5))
## Gaussian pulses
axes[0].plot(t,fA,label='$f_1$', color=PALETTE[0])
axes[0].plot(t,fB,label='$f_2$', color=PALETTE[1])
axes[0].legend(loc='upper center',frameon=False)
axes[0].set_xlim(0,1)
axes[0].set_ylim(0,1.05)
axes[0].set_xlabel(r'$t$ $[\si{\second}]$')
axes[0].set_ylabel(r'$f$ $[\textup{arb. unit}]$')
axes[0].yaxis.set_major_locator(matplotlib.ticker.MaxNLocator(prune='lower',nbins=6))
## Cross-correlation
axes[1].plot(lag, cc, color=PALETTE[0])
axes[1].axvline(x=0, ls=':', color='k',lw=1)
axes[1].set_xlabel(r'$\tau$ $[\si{\second}]$')
axes[1].set_ylabel(r'$\mathcal{C}_{12}$',)
axes[1].set_xlim(-1,1)
axes[1].set_ylim(0,1.05)
axes[1].yaxis.set_major_locator(matplotlib.ticker.MaxNLocator(nbins=6,prune='lower'))
plt.subplots_adjust(wspace=0.3)
tfig = ch3_timelag.save_figure('cross-correlation-example', fext='.pgf')
tfig.caption = r'The left panel shows two Gaussian signals $f_1$ and $f_2$ with peaks at \SI{0.25}{\second} (blue) and \SI{0.75}{\second} (orange), respectively. The right panel shows the cross-correlation $\crosscor[12]$ between $f_1$ and $f_2$ as a function of the offset $\tau$. The dotted black line denotes $\tau=0$ \si{\second}. Note that $\crosscor[12]$ peaks at $\tau=0.5$ \si{\second}, the separation in $t$ between the peaks of $f_1$ and $f_2$.'
\end{pycode}
\py[chapter3timelag]|tfig|
% spell-checker: enable %

A simple example of the cross-correlation is illustrated in \autoref{fig:cross-correlation-example} for two Gaussian pulses, $f_1$ and $f_2$. The right panel shows the two pulses as a function of $t$, with $f_1$ and $f_2$ peaking at \SI{0.25}{\second} and \SI{0.75}{\second}, respectively. The right panel shows the cross-correlation, $\crosscor[12]$, as computed by \autoref{eq:cross_correlation_final}. $\crosscor[12]$ peaks at $\tau=0.5$ \si{\second}, the separation in $t$ between the peaks of $f_1$ and $f_2$. For a more conceptual understanding of $\crosscor[12]$, consider fixing $f_2$ and shifting $f_1$ forward and backward in $t$. The degree to which $f_1$ and $f_2$ overlap or the ``similarity'' between the two signals is analogous to $\crosscor[12]$. $\crosscor[12]$ peaks at $\tau=0.5$ \si{\second} because $f_1$ and $f_2$ are most overlapping when $f_1$ is shifted by \SI{0.5}{\second} relative to its initial position. Throughout the rest of this thesis, I will refer to the value of $\tau$ which maximizes the cross-correlation as the \textit{time lag}. The time lag can be defined mathematically as,
\begin{equation}\label{eq:timelag}
    \tau_{AB} = \argmax_{\tau}\crosscor.
\end{equation}
In this example, the time lag between $f_1$ and $f_2$ is $\tau_{12}=\SI{0.5}{\second}$. Note that for $\crosscor[21]$ (i.e. $f_1$ and $f_2$ reversed), the time lag would be $\tau_{21}=\SI{-0.5}{\second}$ as $f_2$ would need to be shifted 0.5 s to the left in order to maximize the similarity between the two signals. Thus, the order of the two signals in \autoref{eq:cross_correlation_final} will determine the sign of the time lag.

\subsection{Time lag between AIA Channel Pairs}\label{subsec:timelag_aia}

Because $\tau_{AB}$ measures the time delay between peaks in different channels and the temperature sensitivity of each channel is approximately known (see \autoref{fig:aia-temperature-response}), the time lag for a channel pair $A,B$ provides a proxy for the loop plasma cooling time between the nominal temperatures of channels $A$ and $B$. For the six channels of interest, there are fifteen possible channel pairs. Recall that changing the order of the channels only changes the sign of the time lag such that $\tau_{AB}=-\tau_{BA}$. By computing the time lag between every possible channel pair, one can understand how the coronal plasma evolves through the passbands of the instrument and, by extension, how the temperature of the plasma is changing.

% spell-checker: disable %
\begin{pycode}[chapter3timelag]
# Run ebtel++ simulation
config = {
    'total_time': 1e4,
    'tau': 1.0,
    'tau_max': 10.0,
    'loop_length': 4e9,
    'saturation_limit': 1,
    'force_single_fluid': False,
    'use_c1_loss_correction': True,
    'use_c1_grav_correction': True,
    'use_power_law_radiative_losses': True,
    'use_flux_limiting': True,
    'calculate_dem': False,
    'save_terms': False,
    'use_adaptive_solver': True,
    'adaptive_solver_error': 1e-6,
    'adaptive_solver_safety': 0.5,
    'c1_cond0': 2.0,
    'c1_rad0': 0.6,
    'helium_to_hydrogen_ratio': 0.075,
    'surface_gravity': 1.0,
    'heating': OrderedDict({
        'partition': 1.,
        'background': 1e-6,
        'events': [{'event': {'rise_start': 0.0,
                              'rise_end': 0.0,
                              'decay_start': 1.0,
                              'decay_end': 1.0, 
                              'magnitude': 0.01}}]
    }),
}
res = run_ebtel(config, EBTEL_DIR)
t = np.arange(res['time'][0].value, res['time'][-1].value, 1)*res['time'].unit
Te = interp1d(res['time'].value, res['electron_temperature'].value)(t.value)*res['electron_temperature'].unit
n = interp1d(res['time'].value, res['density'].value)(t.value)*res['density'].unit
# Plot normalized intensities for each AIA channel
fig = plt.figure(figsize=texfigure.figsize(pytex,scale=1,height_ratio=1))
ax = fig.add_subplot(211)
aia = InstrumentSDOAIA([0,1]*u.s,None)
intensity = {}
for i,c in enumerate(aia.channels):
    K = splev(Te.value, c['temperature_response_spline'])
    I = config['loop_length']*K*(n**2)
    intensity[c['name']] = I
    ax.plot(t, I/I.max(),
            label=f'{c["wavelength"].value:.0f} '+ r'$\si{\angstrom}$',
            color=PALETTE[i])
ax2 = ax.twinx()
ax2.plot(t, Te.to(u.MK), color='k', alpha=0.5)
# Limits
ax.set_xlim(1,3.25e3)
ax.set_ylim(0,1.05)
ax2.set_ylim(0.1,5.5)
# Labels
ax.set_xlabel(r'$t$ $[\si{\second}]$')
ax.set_ylabel(r'$I_c/\max{I_c}$')
ax2.set_ylabel(r'$T_e$ $[\si{\mega\kelvin}]$')
ax.legend(loc=1,frameon=False)
# Plot cross-correlation
ax = fig.add_subplot(212)
pairs = [
    (94,171),
    (94,211),
    (335,171),
    (211,193),
    (193,171),
    (171,131),
]
for i,(a,b) in enumerate(pairs):
    tl,cc = cross_correlation(t, intensity[f'{a}'], intensity[f'{b}'])
    ax.plot(tl, cc, label=f'{a},{b}', color=PALETTE[i])
    ax.plot(tl[np.argmax(cc)], cc.max(), marker='o', ls='', color=PALETTE[i])
    ax.vlines(tl[np.argmax(cc)].value, -1, cc.max(), linestyles='dotted', lw=1, color=PALETTE[i])
ax.axvline(x=0,ls=':',color='k',lw=1)
ax.set_xlim(-2.5e3,2.5e3)
ax.set_ylim(-0.2,1.05)
ax.set_xlabel(r'$\tau$ $[\si{\second}]$')
ax.set_ylabel(r'$\mathcal{C}_{AB}$')
ax.legend(loc=2,frameon=False,ncol=2)
plt.subplots_adjust(hspace=0.3)
# Save
tfig = ch3_timelag.save_figure('aia-timelags', fext='.pgf')
tfig.placement = '!h'
tfig.caption = r'\textit{Top panel:} Simulated light curves (normalized to the peak value) for the six EUV channels of AIA (left axis) and electron temperature (black line, right axis) for a loop of half-length \SI{40}{\mega\m} cooling from $\approx\SI{5}{\mega\kelvin}$ to $\approx\SI{0.2}{\mega\kelvin}$. The hydrodynamic evolution of the loop was simulated using the EBTEL model and the six light curves were computed using $T_e$ and $n_e$ from the simulation. \textit{Bottom panel:} Cross-correlation as a function of temporal shift, $\tau$, computed from the light curves shown in the top panel for six selected channel pairs. The dotted black line indicates a temporal shift of $\tau=\SI{0}{\second}$. The dotted lines and dots at the peak of each curve denote the time lag for that channel pair.'
\end{pycode}
\py[chapter3timelag]|tfig|
% spell-checker: enable %

An example for a set of simulated AIA light curves is shown in \autoref{fig:aia-timelags}. A loop of half-length \SI{40}{\mega\m} cooling from $\approx\SI{5}{\mega\kelvin}$ down to $\approx\SI{0.2}{\mega\kelvin}$ is simulated using the EBTEL model (see \autoref{sec:ebtel}) and intensities from each of the six AIA channels are computed from \autoref{eq:aia_intensity} using the tabulated temperature response functions. The top panel shows the light curve for each channel (left axis) and the electron temperature (right axis, black). Note that the intensity peaks in successively ``cooler'' channels as the loop cools into and out of the temperature bandpass of each channel (see \autoref{fig:aia-temperature-response}).The bottom panel shows the cross-correlation as a function of temporal offset, $\tau$, for six different pairs of light curves. The time lag, $\tau_{AB}$, for each cross-correlation is denoted by the dotted line and the dot at the peak of the curve. The 94,171 \si{\angstrom} pair has the longest time lag, \SI{> 2e3}{\second}, because the intensity of the \SI{94}{\angstrom} channel, which is most sensitive to $\sim\SI{8}{\mega\kelvin}$ plasma, peaks at $t=\SI{0}{\second}$ while the intensity of the \SI{171}{\angstrom} channel, which is most sensitive to $\sim\SI{1}{\mega\kelvin}$ plasma, does not peak until $t\approx\SI{2.4e3}{\second}$. Conversely, the 171,131 \si{\angstrom} channel pair has the shortest time lag because the \SI{171}{\angstrom} and the cool component of the \SI{131}{\angstrom} temperature response function are heavily overlapping (see \autoref{fig:aia-temperature-response}). 

Note that for a loop which is purely cooling, as is the case in \autoref{fig:aia-timelags}, all of the time lags are positive. Using the convention of \citet{viall_evidence_2012}, ``hotter'' channel is first and the ``cooler'' channel is second in the channel pair such that \textit{a positive time lag implies cooling plasma}. In other words, if the hot emission precedes the cooler emission, the hotter light curve will have to be shifted in the positive direction in $t$ to maximize $\crosscor$. However, negative time lags may also be produced by cooling plasma if a channel is sensitive to both hot and cool plasma (i.e double-peaked in $T_e$, see \autoref{fig:aia-temperature-response}). Additionally, note that a zero time lag, $\tau_{AB}=\SI{0}{\second}$, does not imply a steady light curve or a steady $T_e$; it only implies that variability in channels $A$ and $B$ is coincident \citep{viall_transition_2015,viall_signatures_2016}.

The time-lag analysis is a powerful technique and has been applied in a number of observation and modeling studies \citep[e.g.][]{winebarger_investigation_2016,winebarger_identifying_2018,lionello_can_2016,froment_long-period_2017}. Using 24 hours of AIA observations of \AR{} NOAA 11082, \citet{viall_evidence_2012} computed time lags in every pixel of the \AR{} for all fifteen channel pairs. By computing $\tau_{AB}$ in each pixel, they built up a map of the cooling pattern across the entire \AR{} and found persistent positive time lags in all channel pairs, indicative of cooling plasma. \citet{viall_survey_2017} extended this same technique to the catalogue of \AR s compiled by \citet{warren_systematic_2012} and found similar results for all fifteen \AR s. Additionally, \citet{bradshaw_patterns_2016} applied the time lag analysis to a set of forward-modeled AIA intensities for a range of different heating models and found that their high- and intermediate-frequency nanoflare simulations were most consistent with observations, suggestive of a range of nanoflare heating frequencies.

Throughout this thesis, I will make frequent use of both the emission measure distribution and time-lag analyses. In \autoref{ch:inferring_hot_plasma}, I predict emission measure distribution for several nanoflare heating scenarios in order to determine the observability of $\emd$ for $T_e>T_{peak}$. Additionally, in \autoref{ch:modeling_observables}, I predict AIA intensities over a whole \AR{} for a range of nanoflare heating frequencies and compute the emission measure slope and time lag. Then in \autoref{ch:classifying_observables}, I classify slopes and time lags calculated from real observations in terms of the heating frequency using machine learning models trained on the predicted diagnostics.

% section on potential pitfalls of the time lag?
%% edge effects of time series / windowing effects
%% time variability from other sources (e.g. field rearranging, rotation)
%% interpretability

% Nomenclature
\nomenclature[z-au]{AU}{astronomical unit}
\nomenclature[z-dem]{DEM}{differential emission measure}
\nomenclature[a-h]{$h$}{Planck constant}
\nomenclature[a-c]{$c$}{speed of light in a vacuum}
\nomenclature[g-nu]{$\nu$}{photon frequency}
\nomenclature[g-lambda]{$\lambda$}{wavelength}
\nomenclature[s-k]{$k$}{ionization stage}
\nomenclature[z-los]{LOS}{line-of-sight}
\nomenclature[z-nei]{NEI}{nonequilibrium ionization}
\nomenclature[a-te]{$T_e$}{electron temperature}
\nomenclature[a-ne]{$n_e$}{electron density}
\nomenclature[a-kb]{$k_B$}{Boltzmann constant}
\nomenclature[x-fourier]{$\mathscr{F}$}{Fourier transform}
\nomenclature[x-crosscorrelation]{$\mathcal{C}$}{cross-correlation}
\nomenclature[z-fft]{FFT}{fast Fourier transform}
\nomenclature[x-arcmin]{\si{\arcminute}}{arcminute}
\nomenclature[x-arcsec]{\si{\arcsecond}}{arcsecond}
\nomenclature[z-euv]{EUV}{extreme ultraviolet}
\nomenclature[z-dn]{DN}{digital number, equivalent to counts}
\nomenclature[z-ssw]{SSW}{SolarSoftware, a suite of IDL tools for analysis of solar data}
