% Text for chapter 4
\chapter{synthesizAR: A Framework for Modeling Optically-thin Emission}

% Figure manager for Chapter 3
% spell-checker: disable %
\begin{pycode}[chapter4]
name = 'chapter4'
ch4 = texfigure.Manager(
    pytex,
    os.path.join('.', name),
    number=4,
    **{k: os.path.join('.', name, v) for k,v in manager_opts.items()}
)
from matplotlib.patches import ConnectionPatch
\end{pycode}
% spell-checker: enable %

In order to accurately predict observed optically-thin emission from the impulsively-heated coronal plasma, one must properly account for the field-aligned hydrodynamic response to the energy deposition (\autoref{ch:loops}), the detailed atomic physics that produces the radiation (\autoref{ch:diagnostics}), and geometric effects due to the integration along the LOS. During the course of my PhD, I have a developed a software framework for modeling optically-thin coronal emission called synthesizAR. synthesizAR includes tools for extrapolating the three-dimensional magnetic field from observed LOS magnetograms, configuring input and reading output from ensembles of field-aligned hydrynamic simulations, and computing projections of the emission along the LOS for arbitrary viewing angles. 

synthesizAR is written entirely in the widely-used and open-source Python programming language and developed openly on GitHub\footnote{The entire source code, including installation instructions and links to documentation, can be found at \href{https://github.com/wtbarnes/synthesizAR}{github.com/wtbarnes/synthesizAR}}. The code is fully-documented, including examples, and also includes a test suite that is executed at every code check-in. In this chapter, I give a detailed description of each step of an example workflow using synthesizAR, including working code examples throughout, for a simple dipolar \AR{} in hydrostatic equilibrium. In \autoref{ch:modeling_observables}, I use the synthesizAR code coupled with the two-fluid EBTEL code (see \autoref{sec:ebtel}) to model time-dependent, multi-wavelength emission from an \AR{} for a range of nanoflare heating frequencies.

\section{Building the Magnetic Skeleton}

% Magnetic field extrapolations

The first step in the forward modeling pipeline is to determine the three-dimensional geometry of each magnetic strand in order to construct the magnetic ``skeleton'' of our model \AR{}. As noted in \autoref{ch:loops}, a field-aligned hydrodynamic model only computes the evolution of the plasma along a single thermally-isolated strand such that the three-dimensional position and orientation of the strand relative to the solar surface must be imposed externally. In this example, I will model the emission from \AR{} NOAA 12733 as observed by SDO/AIA on 2019 January 24. \autoref{fig:noaa12733-magnetogram} shows the LOS magnetogram as observed by SDO/HMI.

% spell-checker: disable %
\begin{pycode}[chapter4]
# Load the magnetogram
magnetogram = Map(
    os.path.join(ch4.data_dir, 'hmi_m_45s_2019_01_24_14_01_30_tai_magnetogram.fits')
).rotate(order=3)
# Define AR FOV
l_corner = SkyCoord(Tx=-142*u.arcsec,Ty=50*u.arcsec,frame=magnetogram.coordinate_frame)
r_corner = SkyCoord(Tx=158*u.arcsec,Ty=350*u.arcsec,frame=magnetogram.coordinate_frame)

# Mask the magnetogram off limb
x, y = np.meshgrid(*[np.arange(v.value) for v in magnetogram.dimensions]) * u.pixel
hpc_coords = magnetogram.pixel_to_world(x, y)
r = np.sqrt(hpc_coords.Tx ** 2 + hpc_coords.Ty ** 2) / magnetogram.rsun_obs
mask = np.ma.masked_greater(r, 1)
m_big = Map(magnetogram.data, magnetogram.meta, mask=mask.mask)

# Setup figure
fig = plt.figure(figsize=texfigure.figsize(pytex,scale=1,height_ratio=1.2/2,))
norm = matplotlib.colors.SymLogNorm(50, vmin=-7.5e2, vmax=7.5e2)

# Plot the first magnetogram
ax1 = fig.add_subplot(121, projection=m_big)
m_big.plot(axes=ax1, cmap='better_RdBu_r', norm=norm, annotate=False,)
lon,lat = ax1.coords[0], ax1.coords[1]
lon.frame.set_linewidth(0)
lat.frame.set_linewidth(0)
lon.set_ticks_visible(False)
lat.set_ticks_visible(False)
lon.set_ticklabel_visible(False)
lat.set_ticklabel_visible(False)
m_big.draw_rectangle(l_corner, r_corner.Tx-l_corner.Tx, r_corner.Ty-l_corner.Ty, color='k', lw=1)
m_big.draw_grid(axes=ax1, color='k', alpha=0.25, lw=1)

# Plot the zoomed-in magnetogram
m_small = magnetogram.submap(l_corner, r_corner)
ax2 = fig.add_subplot(122, projection=m_small)
im = m_small.plot(axes=ax2, norm=norm, cmap='better_RdBu_r',annotate=False,)
lon, lat = ax2.coords[0], ax2.coords[1]
lon.frame.set_linewidth(1)
lat.frame.set_linewidth(1)
lon.set_ticklabel()
lat.set_ticklabel(rotation='vertical',)
lon.set_axislabel('Helioprojective Longitude')
lat.set_axislabel('Helioprojective Latitude')
lat.set_axislabel_position('r')
lat.set_ticks_position('r')
lat.set_ticklabel_position('r')

# Add connectors
xpix, ypix = m_big.world_to_pixel(r_corner)
con1 = ConnectionPatch(
    (0,1), (xpix.value, ypix.value), 'axes fraction', 'data', axesA=ax2, axesB=ax1,
    arrowstyle='-', color='k', lw=1
)
xpix, ypix = m_big.world_to_pixel(SkyCoord(r_corner.Tx, l_corner.Ty, frame=m_big.coordinate_frame))
con2 = ConnectionPatch(
    (0,0), (xpix.value,ypix.value), 'axes fraction', 'data', axesA=ax2, axesB=ax1,
    arrowstyle='-', color='k', lw=1
)
ax2.add_artist(con1)
ax2.add_artist(con2)

# Add colorbar
pos = ax2.get_position().get_points()
cax = fig.add_axes([
    pos[0,0], pos[1,1]+0.01, pos[1,0]-pos[0,0], 0.025
])
cbar = fig.colorbar(im, cax=cax, orientation='horizontal')
cbar.locator = matplotlib.ticker.FixedLocator([-1e2,0,1e2])
cbar.update_ticks()
cbar.ax.xaxis.set_ticks_position('top')

# Save figure
tfig = ch4.save_figure('noaa12733-magnetogram', fext='.pdf')
tfig.caption = r'LOS magnetogram of NOAA 12733'
\end{pycode}
\py[chapter4]|tfig|
% spell-checker: enable %

\subsection{Potential Field Extrapolation}\label{sec:potential_field}

As discussed in \autoref{sec:field_extrapolation}, a potential field extrapolation provides a reasonable approximation of the lowest energy configuration of the coronal magnetic field and can be computed relatively efficiently given an input LOS photospheric magnetogram as the lower boundary.

% Schmidt/Sakurai Greens function method

\subsection{Tracing Magnetic Fieldlines}

\subsection{Coordinate Systems for Solar Physics}

\section{Field-aligned Modeling}

% interface configurable; use Martens
% run loop calculation
% show sample loop results for subset of loops

\section{Atomic Physics}

% Choose Fe XVIII (94), XVIII (131) XIV (211), and XVI (335)

\section{Instrument Effects}

\subsection{LOS Projection}
