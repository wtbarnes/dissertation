% Text for chapter 4
\chapter{synthesizAR: A Framework for Modeling Optically-thin Emission}

% Figure manager for Chapter 3
% spell-checker: disable %
\begin{pycode}[chapter4]
name = 'chapter4'
ch4 = texfigure.Manager(
    pytex,
    os.path.join('.', name),
    number=4,
    **{k: os.path.join('.', name, v) for k,v in manager_opts.items()}
)
\end{pycode}
% spell-checker: enable %

In order to accurately predict observed optically-thin emission from the impulsively-heated coronal plasma, one must properly account for the field-aligned hydrodynamic response to the energy deposition (\autoref{ch:loops}), the detailed atomic physics that produces the radiation (\autoref{ch:diagnostics}), and geometric effects due to the integration along the LOS. During the course of my PhD, I have a developed synthesizAR, a software framework for modeling optically-thin coronal emission from magnetic field extrapolations combined with ensembles of field-aligned hydrynamic models. synthesizAR includes tools for extrapolating the three-dimensional magnetic field from observed LOS magnetograms, configuring input and reading output from field-aligned hydrodynamic simulations, and computing projections for arbitrary viewing angles. synthesizAR is written entirely in the widely-used Python programming language and developed openly on GitHub\footnote{The entire source code, including installation instructions and links to the documentation, can be found at \href{https://github.com/wtbarnes/synthesizAR}{github.com/wtbarnes/synthesizAR}}. The code is fully-documented, including examples, and also includes a test suite that is executed at every code check-in. In this chapter, I give a detailed description of each step of an example workflow using synthesizAR, including working code examples throughout, for a simple dipolar \AR{} in hydrostatic equilibrium.

\section{Magnetic Field Geometry}

% Magnetic field extrapolations
% Tracing strands
% Loop coordinate systems

\section{Field-aligned Modeling}

% interface configurable; use Martens
% run loop calculation
% show sample loop results for subset of loops

\section{Atomic Physics}

% Choose Fe XVIII (94), XVIII (131) XIV (211), and XVI (335)

\section{Instrument Effects}

\section{Projecting along the Line-of-Sight}
