% Text for chapter 8
\chapter{Future Work and Conclusions}\label{ch:conclusions}

% spell-checker: disable %
\begin{pycode}[chapter8]
name = 'chapter8'
ch8 = texfigure.Manager(
    pytex,
    os.path.join('.', name),
    number=8,
    **{k: os.path.join('.', name, v) for k,v in manager_opts.items()}
)
from script import spatial_heating_profile
\end{pycode}
% spell-checker: enable %

\section{Conclusions}\label{sec:conclusions}

% outline three results sections and put the bullets there, just paraphrase from those results sections

\section{Future Work}\label{sec:future-work}

\subsection{Bundle Heating Model}\label{sec:bundle}

% spell-checker: disable %
\begin{pycode}[chapter8]
channel_pairs = [(94,335), (335,171), (211,193), (171,131)]
correlation_threshold = 0.1
norm = matplotlib.colors.Normalize(vmin=-(5e3*u.s).to(u.s).value,
                                   vmax=(5e3*u.s).to(u.s).value)
plot_params = { 'title': False, 'annotate': False, 'norm': norm, 'cmap': 'idl_bgry_004',}

fig = plt.figure(figsize=texfigure.figsize(
    pytex,
    scale=1,
    height_ratio=1,
))

for i,(ca,cb) in enumerate(channel_pairs):
    m = Map(ch8.data_file(os.path.join('bundle-model', f'timelag_{ca}_{cb}.fits')))
    mc = Map(ch8.data_file(os.path.join('bundle-model', f'correlation_{ca}_{cb}.fits')))
    m = Map(m.data, m.meta, mask=np.where(mc.data<=correlation_threshold, True, False))
    m = m.submap(SkyCoord(Tx=-440*u.arcsec,Ty=-380*u.arcsec,frame=m.coordinate_frame),
                 SkyCoord(Tx=-185*u.arcsec,Ty=-125*u.arcsec,frame=m.coordinate_frame))
    ax = fig.add_subplot(2, 2, i+1, projection=m)
    im = m.plot(axes=ax, **plot_params)
    ax.grid(alpha=0)
    lon = ax.coords[0]
    lat = ax.coords[1]
    if i == 2:
        lat.set_axislabel(r'Helioprojective Latitude',)
        lat.set_ticklabel(rotation='vertical',exclude_overlapping=True)
        lon.set_axislabel(r'Helioprojective Longitude',)
        lon.set_ticklabel(exclude_overlapping=True)
    else:
        lat.set_ticklabel_visible(False)
        lon.set_ticklabel_visible(False)
    xtext,ytext = m.world_to_pixel(SkyCoord(-435*u.arcsec, -130*u.arcsec, frame=m.coordinate_frame))
    ax.text(xtext.value, ytext.value, r'{}-{} $\si{{\angstrom}}$'.format(ca, cb),
            color='k', fontsize=plt.rcParams['legend.fontsize'],
            verticalalignment='top', horizontalalignment='left')
    if i == 0:
        pos = ax.get_position().get_points()
        cbar_x = pos[0,0]
        cbar_y = pos[1,1]+0.01
    if i == 1:
        pos = ax.get_position().get_points()
        cbar_w = pos[1,0] - cbar_x 

# Colorbar
cax = fig.add_axes([cbar_x, cbar_y, cbar_w, 0.02])
cbar = fig.colorbar(im, cax=cax, orientation='horizontal')
cbar.ax.xaxis.set_ticks_position('top')
cbar.ax.tick_params(width=0.5)
cbar.outline.set_linewidth(0.5)

plt.subplots_adjust(wspace=0.025,hspace=0.025)

# Save
tfig = ch8.save_figure('bundle-timelags', fext='.pdf')
tfig.caption = r"Timelag maps produced by the bundle heating model as simulated from a field extrapolation of \AR{} NOAA 1158. A sample of four channel pairs are shown here: 94-335, 335-171, 211-193, and 171-131 \si{\angstrom}. The value of each pixel indicates the temporal offset, in seconds, which maximizes the cross-correlation (see \autoref{eq:timelag}). The range of the colorbar is $\pm\SI{5000}{\second}$. If $\max{\mathcal{C}_{AB}}<0.1$, the pixel is masked and colored white."
\end{pycode}
\py[chapter8]|tfig|
% spell-checker: enable %

\subsection{Thermal Non-equilibrium}\label{sec:tne}

% spell-checker: disable %
\begin{pycode}[chapter8]
# Read in data
i_start,i_stop,i_step = 0,-1,10
with h5py.File(os.path.join(ch8.data_dir, 'hydrad_tne_results.h5'), 'r') as hf:
    t_grid = u.Quantity(hf['time'][i_start:i_stop:i_step],hf['time'].attrs['unit'])
    s_grid = u.Quantity(hf['coordinate'],hf['coordinate'].attrs['unit'])
    Te_grid = u.Quantity(hf['electron_temperature'][i_start:i_stop:i_step,:],
                         hf['electron_temperature'].attrs['unit'])
    Ti_grid = u.Quantity(hf['ion_temperature'][i_start:i_stop:i_step,:],
                         hf['ion_temperature'].attrs['unit'])
    n_grid = u.Quantity(hf['density'][i_start:i_stop:i_step,:],
                        hf['density'].attrs['unit'])
    v_grid = u.Quantity(hf['velocity'][i_start:i_stop:i_step,:],
                        hf['velocity'].attrs['unit'])
\end{pycode}
% spell-checker: enable %

% spell-checker: disable %
\begin{pycode}[chapter8]
h_profile = spatial_heating_profile(s_grid, os.path.join(ch8.data_dir, 'hydrad_tne_config.asdf'))
fig = plt.figure(figsize=texfigure.figsize(
    pytex,
    scale=0.65,
    height_ratio=1,
))
ax = fig.gca()
ax.plot(s_grid.to(u.Mm), h_profile)
ax.set_xlim(s_grid.to(u.Mm)[[0,-1]].value)
ax.set_ylim(0,5.1e-3)
ax.ticklabel_format(axis='y', style='sci', scilimits=(0,0))
ax.set_xlabel(r'$s$ $[\si{\mega\m}]$')
ax.set_ylabel(r'$E_H$ $[\si{\erg\per\cubic\cm\per\second}]$')
tfig = ch8.save_figure('future-work:tne-heating', fext='.pgf')
tfig.caption = r'Heating input as a function of field-aligned coordinate, $s$, for simulating TNE using the HYDRAD code. The full-length of the strand is \SI{120}{\mega\m}. The total heating profile is a combination of two Gaussian heating profiles of width \SI{10}{\mega\m} at the two footpoints, $s=\SI{5}{\mega\m}$ and $s=\SI{115}{\mega\m}$. The heating rates of both pulses \SI{5e-3}{\erg\per\cubic\cm\per\second}. The heating is turned on at $t=\SI{0}{\second}$ and is kept constant for the entire simulation.'
tfig.figure_width = r'0.65\textwidth'
\end{pycode}
\py[chapter8]|tfig|
% spell-checker: enable %

% spell-checker: disable %
\begin{pycode}[chapter8]
# Setup grid
tmesh, smesh = np.meshgrid(s_grid.to(u.Mm), t_grid.to(u.h))
# Setup figure
fig,axes = plt.subplots(2,1,figsize=texfigure.figsize(
    pytex,
    scale=1,
    height_ratio=2/3,
    figure_width_context='figurewidth',)
)
# Temperature
im_T = axes[0].pcolormesh(
    smesh, tmesh, Te_grid.to(u.MK),
    cmap='plasma',
    vmin=0.1,
    vmax=3,
    rasterized=True)
# Density
im_n = axes[1].pcolormesh(
    smesh, tmesh, n_grid,
    cmap='plasma',
    norm=matplotlib.colors.LogNorm(vmin=1e9,vmax=1e10),
    rasterized=True)
# Labels, ticks
axes[0].xaxis.set_major_locator(matplotlib.ticker.NullLocator())
axes[0].yaxis.set_major_locator(matplotlib.ticker.MaxNLocator(nbins=4, prune='lower'))
axes[1].yaxis.set_major_locator(matplotlib.ticker.MaxNLocator(nbins=4,))
axes[1].set_xlabel(r'$t$ $[\si{\hour}]$')
axes[0].set_ylabel(r'$s$ $[\si{\mega\m}]$')
axes[1].set_ylabel(r'$s$ $[\si{\mega\m}]$')
# Colorbars
plt.subplots_adjust(hspace=0.05)
pos = axes[0].get_position().get_points()
cax = fig.add_axes([pos[1,0]+0.005,pos[0,1],0.015,pos[1,1]-pos[0,1]])
cbar = fig.colorbar(im_T, cax=cax, orientation='vertical')
pos = axes[1].get_position().get_points()
cax = fig.add_axes([pos[1,0]+0.005,pos[0,1],0.015,pos[1,1]-pos[0,1]])
cbar = fig.colorbar(im_n, cax=cax, orientation='vertical')
# Save figure
tfig = ch8.save_figure('future-work:tne-results', fext='.pdf')
tfig.caption = r'Electron temperature, in \si{\mega\kelvin} (top), and density, in \si{\per\cubic\cm} (bottom), as a function of field-aligned coordinate, $s$, and time, $t$ as simulated by the HYDRAD code for a semi-circular loop of full-length $L=\SI{120}{\mega\m}$. The time-independent heating function is localized to the footpoints and is shown as a function of $s$ in \autoref{fig:future-work:tne-heating}.'
\end{pycode}
\py[chapter8]|tfig|
% spell-checker: enable %
