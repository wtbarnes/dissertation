\begin{abstract}
While it is generally agreed that the continually stressed coronal
magnetic field is responsible for the million-degree temperatures in the
upper solar atmosphere, the mechanism responsible for transporting this
stored energy to the coronal plasma is yet unknown. In particular, the
frequency with which this energy is deposited into active regions, areas
of intense magnetic activity, remains largely unconstrained. In this
dissertation, I use a new forward modeling pipeline, combined with
well-known machine learning classification methods, to better constrain
the frequency of energy deposition in active region cores. Specifically,
I simulate time-dependent, multi-wavelength emission over an entire
active region for a range of nanoflare frequencies. From these synthetic
intensities, I compute two commonly-used observables, the timelag and
the emission measure slope, and compare the distributions of these
parameters in order to better understand how the underlying heating
frequency is related to observable quantities. I then use these
synthetic data to train a random forest classifier in order to classify
real data in terms of the heating frequency. In doing so, I am able to
map the heating frequency, pixel by pixel, across an entire active
region. In addition to this parameter survey of heating frequencies, I
also describe a model for relating the underlying magnetic field
structure of the active region to the heating frequency and compute the
resulting emission in multiple narrowband channels. The resulting
observables from this new model are assessed against the classified
observations. Altogether, this work represents a critical step in
constraining heating frequency in active region cores.

A novel component of this dissertation is the modular and scalable
fowarding modeling pipeline, written in the Python programming language,
that builds a ``magnetic skeleton'' from a three-dimensional field
extrapolation, configures thousands of field-aligned hydrodynamic loop
models, and computes arbitrary line-of-sight projections of the
time-dependent, three-dimensional active region emission. Another novel
component of this dissertation is the use of a random forest
classification model. By training such a model on our synthetic data, we
are able to systematically and quantitatively assess agreement between
modeled nanoflare frequencies and observations.
\end{abstract}
