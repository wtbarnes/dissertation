\begin{abstract}
    The solar corona, the outermost layer of the Sun's atmosphere, is heated to temperatures in excess of one million Kelvin, nearly three orders of magnitude greater than the surface of the Sun. While it is generally agreed that the continually stressed coronal magnetic field plays a role in producing these million-degree temperatures, the exact mechanism responsible for transporting this stored energy to the coronal plasma is yet unknown. Nanoflares, small-scale bursts of energy, have long been proposed as a candidate for heating the non-flaring corona, especially in areas of high magnetic activity. However, a direct detection of heating by nanoflares has proved difficult and as such, properties of this proposed heating mechanism remain largely unconstrained. In this thesis, I use a hydrodynamic model of the coronal plasma combined with a sophisticated forward modeling approach and machine learning classification techniques to predict signatures of nanoflare heating and compare these predictions to real observational data. In particular, the focus of this work is constraining the frequency with which nanoflares occur on a given magnetic field line in non-flaring active regions. 
    
    First, I give an introduction to the structure of the solar atmosphere and coronal heating, discuss the hydrodynamics of coronal loops, and provide an overview of the important emission mechanisms in a high-temperature, optically-thin plasma. Then, I describe the forward modeling pipeline for predicting time-dependent, multi-wavelength emission over an entire active region. Next, I use a hydrodynamic model of a single coronal loop to predict signatures of ``very hot'' plasma produced by nanoflares and find that several effects are likely to affect the observability of this direct signature of nanoflare heating. Then, I use the forward modeling code described above to simulate time-dependent, multi-wavelength AIA emission from active region NOAA 1158 for a range of nanoflare frequencies and find that signatures of the heating frequency persist in multiple observables. Finally, I use these predicted diagnostics to train a random forest classifier and apply this model to real AIA observations of NOAA 1158. Altogether, this thesis represents a critical step in systematically constraining the frequency of energy deposition in active regions.
\end{abstract}
