\begin{abstract}
The solar corona, the outermost layer of the Sun's atmosphere, is heated to temperatures in excess of one million Kelvin, nearly three orders of magnitude greater than the surface of the Sun. This so-called ``coronal heating problem'' has occupied the field of solar astrophysics for over seventy years and is one of the most important open questions in astronomy as a whole. While it is generally agreed that the continually stressed coronal magnetic field plays a role in producing these million-degree temperatures, the exact mechanism responsible for transporting this stored energy to the coronal plasma is yet unknown. Nanoflares, small-scale bursts of energy likely resulting from the frequent reconnection of twisted magnetic field lines, have long been proposed as a candidate for heating the non-flaring corona, especially in areas of high magnetic activity. However, a direct detection of heating by nanoflares has proved difficult due to their faint, transient nature and as such, properties of this proposed heating mechanism remain largely unconstrained. In this thesis, I use a hydrodynamic model of the coronal plasma combined with a sophisticated forward modeling approach and machine learning classification techniques to predict signatures of nanoflare heating and compare these predictions to real observational data. In particular, the focus of this work is constraining the frequency with which nanoflares occur on a given magnetic field line in non-flaring active regions. 

First, I give an introduction to the structure of the solar atmosphere and coronal heating, discuss the hydrodynamics of coronal loops, and provide an overview of the important emission mechanisms in a high-temperature, optically-thin plasma. Then, I describe the forward modeling pipeline for predicting time-dependent, multi-wavelength emission over an entire active region. Next, I use a hydrodynamic model of a single coronal loop to predict signatures of ``very hot'' plasma produced by single nanoflares. I find that several effects, including flux limiting, nonequilibrium ionization, and nanoflare duration, are likely to affect the observability of this direct signature of nanoflare heating. Then, I use the forward modeling code described above to simulate time-dependent, multi-wavelength AIA emission from active region NOAA 1158 for a range of nanoflare frequencies and find that signatures of the heating frequency persist in multiple observable quantities. Finally, I use these predicted diagnostics to train a random forest classifier and apply this model to real AIA observations of NOAA 1158. In doing so, I am able to map the heating frequency, pixel by pixel, across the entire active region. Altogether, this thesis represents a critical step in systematically constraining the frequency of energy deposition in active regions.

A novel component of this thesis is the development of a modular forward modeling pipeline, written in the Python programming language, that builds a ``magnetic skeleton'' from a three-dimensional field extrapolation, configures thousands of field-aligned hydrodynamic loop models, and computes arbitrary line-of-sight projections of the time-dependent, three-dimensional active region emission. The code is flexible and scalable and is openly-developed such that it may be used and improved by the larger solar physics community. Another novel component of this thesis is the use of machine learning to compare real observations and model results. By training a random forest classifier on predicted diagnostics, I am able to systematically and quantitatively assess observations in the context of multiple diagnostics in order to make an accurate prediction of the properties of the heating.
\end{abstract}
