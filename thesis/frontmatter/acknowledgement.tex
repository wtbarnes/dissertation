% ************************** Thesis Acknowledgements **************************
\begin{acknowledgements}      
This work would not have been possible, or at the very least far less enjoyable, without the help and support of my supervisors, colleagues, friends and family. 

First, I would like to thank my thesis committee members, Dr. David Alexander and Dr. Maarten de Hoop, for agreeing to serve on my committee and reading a first draft of this work. I would especially like to thank Dr. Alexander for his advice and guidance, both career- and research-related, during my time as a graduate student and for helping me navigate the field of solar physics.

As a graduate student, it has been my great pleasure to work with and be advised by Dr. Stephen Bradshaw. I have benefited immensely from his vast knowledge of field-aligned hydrodynamics and atomic physics as well as his careful and measured approach to research. Most importantly, he has taught me how to be an independent researcher and I am extremely grateful for his mentorship and friendship during my time at Rice.

I also owe a special debt of gratitude to my undergraduate research advisor, Dr. Lorin Matthews (Baylor University), for teaching me about the microphysics of astrophysical dusty plasmas and for inspiring me to go to graduate school. I am grateful for her patience and kindness as a mentor early in my physics education.

During my brief time in the solar physics community, I have been fortunate to collaborate with several talented and accomplished researchers. I am extremely grateful to Professor Peter Cargill (Imperial College London, University of St Andrews) for sharing his unparalleled knowledge of coronal loop physics and for his patience in guiding me through the writing and publication of two papers early in my graduate career, the first of which comprises \autoref{ch:inferring_hot_plasma} of this thesis. Additionally, I would like to thank Dr. Nicholeen Viall (NASA Goddard Space Flight Center) for lending her observational expertise and detailed knowledge of the temperature sensitivity of the AIA passbands and for providing helpful comments and revisions on \autoref{ch:modeling-observables} and \autoref{ch:classifying-observables} of this thesis. I would also like to thank Dr. Jim Klimchuk, Dr. Harry Warren, Dr. Jeffrey Reep, Dr. Jack Ireland, and Dr. Ken Dere.

I am extremely indebted to the members of the SunPy community for volunteering their time and effort to build a sustainable software ecosystem for solar physics. In particular, I would like to thank Dr. Stuart Mumford for his tireless and often thankless efforts to continually improve and develop SunPy and for always having the answer to any question related to Python or solar coordinate systems.

I am very grateful to the many people at Rice and in Houston who made my time as a graduate student all the more enjoyable. Many thanks go to Dan, Kong, Joe, Nathan, Loah, Brandon, Laura, Alison, Alex, and Shah for hearing my complaints at lunch, sharing more than a few beers at Valhalla, and making graduate school bearable and, on occasion, fun. I would especially like to thank Joe and Mitch for their friendship and support over the last decade, both in Waco and in Houston.

I would like to thank my parents, Mark and Terri, for their financial, emotional, and physical support throughout my entire life, across multiple states and even a few continents. I would also like to thank my siblings, Jessie and Wesley, for always being willing to remind me that I am not that smart. I owe special thanks to my in-laws, Jim and Susan, as well as to my siblings-in-law, Tara and Michael, for the many rounds of disc golf and even more free meals; and Tamara and Mike for making me feel welcome when I first moved to Houston and for continuing to support me as I prepare to leave.

Lastly, I am grateful to my wife Morgan.
\end{acknowledgements}
